% Options for packages loaded elsewhere
\PassOptionsToPackage{unicode}{hyperref}
\PassOptionsToPackage{hyphens}{url}
%
\documentclass[
]{article}
\usepackage{amsmath,amssymb}
\usepackage{lmodern}
\usepackage{iftex}
\ifPDFTeX
  \usepackage[T1]{fontenc}
  \usepackage[utf8]{inputenc}
  \usepackage{textcomp} % provide euro and other symbols
\else % if luatex or xetex
  \usepackage{unicode-math}
  \defaultfontfeatures{Scale=MatchLowercase}
  \defaultfontfeatures[\rmfamily]{Ligatures=TeX,Scale=1}
\fi
% Use upquote if available, for straight quotes in verbatim environments
\IfFileExists{upquote.sty}{\usepackage{upquote}}{}
\IfFileExists{microtype.sty}{% use microtype if available
  \usepackage[]{microtype}
  \UseMicrotypeSet[protrusion]{basicmath} % disable protrusion for tt fonts
}{}
\makeatletter
\@ifundefined{KOMAClassName}{% if non-KOMA class
  \IfFileExists{parskip.sty}{%
    \usepackage{parskip}
  }{% else
    \setlength{\parindent}{0pt}
    \setlength{\parskip}{6pt plus 2pt minus 1pt}}
}{% if KOMA class
  \KOMAoptions{parskip=half}}
\makeatother
\usepackage{xcolor}
\IfFileExists{xurl.sty}{\usepackage{xurl}}{} % add URL line breaks if available
\IfFileExists{bookmark.sty}{\usepackage{bookmark}}{\usepackage{hyperref}}
\hypersetup{
  pdftitle={Space for WHAM: a multi-region, multi-stock generalization of the Woods Hole Assessment Model with an application to black sea bass},
  pdfauthor={Timothy J. Miller1; Kierstin Curti; Alex Hansell},
  hidelinks,
  pdfcreator={LaTeX via pandoc}}
\urlstyle{same} % disable monospaced font for URLs
\usepackage[margin=1in]{geometry}
\usepackage{graphicx}
\makeatletter
\def\maxwidth{\ifdim\Gin@nat@width>\linewidth\linewidth\else\Gin@nat@width\fi}
\def\maxheight{\ifdim\Gin@nat@height>\textheight\textheight\else\Gin@nat@height\fi}
\makeatother
% Scale images if necessary, so that they will not overflow the page
% margins by default, and it is still possible to overwrite the defaults
% using explicit options in \includegraphics[width, height, ...]{}
\setkeys{Gin}{width=\maxwidth,height=\maxheight,keepaspectratio}
% Set default figure placement to htbp
\makeatletter
\def\fps@figure{htbp}
\makeatother
\setlength{\emergencystretch}{3em} % prevent overfull lines
\providecommand{\tightlist}{%
  \setlength{\itemsep}{0pt}\setlength{\parskip}{0pt}}
\setcounter{secnumdepth}{5}
\newlength{\cslhangindent}
\setlength{\cslhangindent}{1.5em}
\newlength{\csllabelwidth}
\setlength{\csllabelwidth}{3em}
\newlength{\cslentryspacingunit} % times entry-spacing
\setlength{\cslentryspacingunit}{\parskip}
\newenvironment{CSLReferences}[2] % #1 hanging-ident, #2 entry spacing
 {% don't indent paragraphs
  \setlength{\parindent}{0pt}
  % turn on hanging indent if param 1 is 1
  \ifodd #1
  \let\oldpar\par
  \def\par{\hangindent=\cslhangindent\oldpar}
  \fi
  % set entry spacing
  \setlength{\parskip}{#2\cslentryspacingunit}
 }%
 {}
\usepackage{calc}
\newcommand{\CSLBlock}[1]{#1\hfill\break}
\newcommand{\CSLLeftMargin}[1]{\parbox[t]{\csllabelwidth}{#1}}
\newcommand{\CSLRightInline}[1]{\parbox[t]{\linewidth - \csllabelwidth}{#1}\break}
\newcommand{\CSLIndent}[1]{\hspace{\cslhangindent}#1}
\usepackage{url}
\usepackage{setspace}
%\singlespacing
%\onehalfspacing
\doublespacing
\usepackage{lineno}
\linenumbers
\usepackage[belowskip=0pt,aboveskip=0pt]{caption}
\usepackage{relsize}
\usepackage{float}
% \usepackage[section]{placeins}
\usepackage{lscape}
\usepackage{longtable}
\usepackage{amsmath,rotating}
\usepackage[scanall]{psfrag}
\usepackage{bm}
\usepackage{caption,graphics}
\usepackage{graphicx}
%\usepackage{natbib}
%\usepackage[nottoc]{tocbibind}
%\usepackage{indentfirst}
\usepackage{sectsty}
\usepackage{color}
\usepackage{fancyhdr}
\usepackage{xspace}
\usepackage{textcomp}
\usepackage{upgreek}

\newcommand{\blandscape}{\begin{landscape}}
\newcommand{\elandscape}{\end{landscape}}
\usepackage{booktabs}
\usepackage{longtable}
\usepackage{array}
\usepackage{multirow}
\usepackage{wrapfig}
\usepackage{float}
\usepackage{colortbl}
\usepackage{pdflscape}
\usepackage{tabu}
\usepackage{threeparttable}
\usepackage{threeparttablex}
\usepackage[normalem]{ulem}
\usepackage{makecell}
\usepackage{xcolor}
\ifLuaTeX
  \usepackage{selnolig}  % disable illegal ligatures
\fi

\title{Space for WHAM: a multi-region, multi-stock generalization of the
Woods Hole Assessment Model with an application to black sea bass}
\author{Timothy J. Miller\textsuperscript{1} \and Kierstin
Curti \and Alex Hansell}
\date{14 November, 2024}

\begin{document}
\maketitle

\(^1\)\href{mailto:timothy.j.miller@noaa.gov}{\nolinkurl{timothy.j.miller@noaa.gov}},
Northeast Fisheries Science Center, National Marine Fisheries Service,
166 Water Street, Woods Hole, MA 02543, USA\\

\pagebreak

\hypertarget{main-message}{%
\subsection*{Main Message}\label{main-message}}
\addcontentsline{toc}{subsection}{Main Message}

Describes the multi-region, multi-stock generalization of WHAM and its
usage to evaluate evidence of alternative hypotheses about temperature
effects on recruitment of black sea bass stock components.

\hypertarget{abstract}{%
\subsection*{Abstract}\label{abstract}}
\addcontentsline{toc}{subsection}{Abstract}

The Woods Hole Assessment Model (WHAM) is a state-space age-structured
assessment model that is used to assess and manage many stocks in the
Northeast US. We first describe a multi-stock, multi-region extension of
WHAM that treats the population and fleet dynamics seasonally and allows
movement by season and region to be functions of time- and age-varying
autocorrelated random effects and environmental covariates. We then
illustrate the model by applying it to data for the northern and
southern components of the NEUS black sea bass stock and evaluate
alternative hypotheses of bottom temperature and random effects on
recruitment and natural mortality. We show strong evidence for
temperature effects on recruitment, primarily for the northern stock
component, and no evidence for including random effects or temperature
effects on age 1 natural mortality.

\pagebreak

\hypertarget{introduction}{%
\section*{Introduction}\label{introduction}}
\addcontentsline{toc}{section}{Introduction}

A state-space statistical approach and maximum marginal likelihood or
Bayesian fitting of stock assessment models allows estimation of time-
and age-varying population attributes as random effects Miller et al.
(2016b). This estimation approach is considered an essential feature of
gold-standard assessment models that we use in tactical management of
commercially important fish stocks (Punt et al. 2020). The State-space
Assessment Model (SAM, Nielsen and Berg 2014) continues to be developed
and remains widely used within ICES to assess European fish stocks.
Various state-space models are being used to manage cod and plaice
stocks in the waters of Eastern Canada (Perreault et al. 2020; Varkey et
al. 2022) and to the south, the Woods Hole Assessment Model (WHAM, Stock
and Miller 2021) is now used to assess many fish stocks in the Northwest
Atlantic Ocean (NEFSC 2022a, 2022b; NEFSC 2024).

WHAM is an R package developed and maintained at NOAA's Northeast
Fisheries Science Center (\url{https://timjmiller.github.io/wham},
Miller and Stock 2020; Stock and Miller 2021). WHAM can be configured to
fit a wide range of age-structured models from traditional statistical
catch-at-age models without any random effects to models with several
time and age varying process errors and possibly effects of
environmental covariates on various demographic parameters. Like SAM,
WHAM models are built using the Template Model Builder package (TMB,
Kristensen et al. 2016) which provides a computationally efficient means
of fitting an extremely wide class of models with random effects. WHAM
has undergone active development since its creation and includes random
effects options for in recruitment, inter-annual transitions numbers at
age (hereafter referred to as ``survival''), fishery and index
selectivity, natural mortality, and catchability.

However, WHAM has up to now only allowed models with one stock
(component) and a single region and without seasonal changes in stock
dynamics. Using such models for stocks that have subcomponents with
varying seasonal movement can provide incorrect inferences and poor
management advice (Ying et al. 2011; Cao et al. 2014; Bosley et al.
2022). Furthermore, the ability to account for spatial structure and
model multiple stocks are also important features of leading-edge
assessment modeling frameworks (Punt et al. 2020). We describe here the
implementation of these features and other extensions since Stock and
Miller (2021) in WHAM version 2.0. Many of these new configuration
options can be useful whether modeling 1 or more stocks and regions.

This extension of WHAM was developed in concert with a stock assessment
for black sea bass through the NEFSC research track assessment process
where new modeling frameworks were recently examined and modeling
multiple stocks or stock components simultaneously was of interest
(NEFSC 2023). \textcolor{red}{[Text on black sea bass here]} We apply
WHAM 2.0 to two stock components of black sea bass off the coast of the
NEUS and evaluate evidence for alternative hypotheses of temporal
variation and effects specifically of bottom temperature on recruitment
and natural mortality of age 1 individuals.

\hypertarget{methods}{%
\section*{Methods}\label{methods}}
\addcontentsline{toc}{section}{Methods}

\hypertarget{wham-description}{%
\subsection*{WHAM description}\label{wham-description}}
\addcontentsline{toc}{subsection}{WHAM description}

Many of the options and equations of WHAM version 2.0 are the same as
those in Stock and Miller (2021), so we will only describe extensions
and differences that have occurred since their first description of
WHAM. The new version of WHAM can model multiple stocks and survival,
movement, harvest and natural mortality are tracked for each stock. Much
of the description below is for a specific stock \(s\), but, for
simplicity, this subscript is implicit except when necessary.

\hypertarget{the-probability-transition-matrix}{%
\subsubsection*{The probability transition
matrix}\label{the-probability-transition-matrix}}
\addcontentsline{toc}{subsubsection}{The probability transition matrix}

Because individuals may be alive in one of several regions or harvested
in one of several fleets, it is helpful to consider these as distinct
categories or states and treat the number of individuals occurring in
each category over time as a multi-state model. Approaches to modeling
transitions among these tates may treat time discretely (e.g., Arnason
1972; Schwarz et al. 1993) or continuously (e.g., Hearn et al. 1987;
Commenges 1999; Andersen and Keiding 2002). Multi-state models can
define a probability transition matrix (PTM) that describes the
probability of individuals occurring in different states at the end of a
time interval \(\delta\), conditional on being in each of those state at
beginning of the interval. For fish populations these states would be
defined as being alive in a particular region or being dead due to
fishing from a particular fleet or natural mortality. The time interval
\(i\) with duration \(\delta_i\) would be a season and the PTMs would be
uniquely defined for each stock by season \(i\), year \(y\), and age
\(a\) on January 1. Each row and column of the PTM correspond to one the
states: alive in region \(r\), dead in fleet \(f\), or dead from natural
causes. The probabilities in each row sum to unity and assume an
individual is in the corresponding state at the beginning of the
interval. Given \(n_R\) regions and \(n_F\) fleets, the square PTM
(\(n_R + n_F + 1\) rows and columns) as a function of sub-matrices is
\begin{equation}\label{eq:ptm}
  \mathbf{P}_{y,a,i} = \begin{bmatrix}
    \mathbf{O}_{y,a,i} & \mathbf{H}_{y,a,i} & \mathbf{D}_{y,a,i} \\
    0 & \mathbf{I}_{H} & 0\\
    0 & 0 & 1
  \end{bmatrix}
\end{equation} where \begin{equation*}
  \mathbf{O}_{y,a,i} = 
  \begin{bmatrix}
    O_{y,a,i}(1,1) & \cdots & O_{y,a,i}(1,n_R) \\
    \vdots & \ddots & \vdots \\
    O_{y,a,i}(n_R,1) & \cdots & O_{y,a,i}(n_R,n_R)
  \end{bmatrix}
\end{equation*} is the \(n_R \times n_R\) matrix defining survival and
occurring in each region at the end of the interval, \begin{equation*} 
  \mathbf{H}_{y,a,i} = 
  \begin{bmatrix}
    H_{y,a,i}(1,1) & \cdots & H_{y,a,i}(1,n_F) \\
    \vdots & \ddots & \vdots \\
    H_{y,a,i}(n_R,1) & \cdots & H_{y,a,i}(n_R,n_F)
  \end{bmatrix}
\end{equation*} is the \(n_R \times n_F\) matrix defining probabilities
of being captured in each fleet during the interval, and
\(\mathbf{D}_{y,a,i}\) is the \(n_R x 1\) matrix of probabilities of
dying due to natural mortality during the interval. We have the identity
matrix \(\mathbf{I}_{H}\) for the states for capture by each fleet and a
1 for the state for natural mortality because the probabilities of being
in one of the mortality states given starting the interval in that state
is unity (no zombies allowed).

WHAM uses these PTMs to model abundance proportions in each state rather
than true probabilities where numbers in each state would be multinomial
distributed as in a model for tagging data where fates of individual
fish are assumed independent. The PTMs determine the expected numbers 1)
in each state on January 1 of year \(t+1\) at age \(a+1\) given the
abundances at age \(a\) on January 1 of year \(t\), 2) captured over the
year in each fleet, 3) available to each index, and 4) alive at the time
and in the region where spawning occurs.

\hypertarget{single-region-ptms}{%
\subsubsection*{Single region PTMs}\label{single-region-ptms}}
\addcontentsline{toc}{subsubsection}{Single region PTMs}

When there is only one region, \begin{equation}\label{eq:ptm_1_region}
\mathbf{P}_{y,a,i} = 
  \begin{bmatrix}
     S_{y,a,i} & \mathbf{H}_{y,a,i} & D_{y,a,i} \\
     0 & \mathbf{I}_{H} & 0\\
     0 & 0 & 1
  \end{bmatrix}
\end{equation} where \(S_{y,a,i} = e^{-Z_{y,a,i}\delta_i}\),
\(\mathbf{H}_{y,a,i}\) is a 1 x \(n_F\) matrix with elements for each
fleet \(f\):
\(\frac{F_{y,a,i,f}}{Z_{y,a,i}}\left(1 - e^{-Z_{y,a,i}\delta_i}\right)\),
\(D_{y,a,i} = \frac{M_{y,a}}{Z_{y,a,i}}\left(1 - e^{-Z_{y,a,i}\delta_i}\right)\),
and \(Z_{y,a,i} = M_{y,a} + \sum^{n_F}_{f=1} F_{y,a,i,f}\) is the total
mortality rate.

\hypertarget{multi-region-ptms}{%
\subsubsection*{Multi-region PTMs}\label{multi-region-ptms}}
\addcontentsline{toc}{subsubsection}{Multi-region PTMs}

When there is more than 1 region, WHAM can model survival and movement
as processes occurring sequentially or simultaneously. The sequential
assumption is used widely in spatially explicit model (e.g., SS3). Under
the sequential assumption, survival and death occur over the interval
and movement among regions occurs instantly at either the beginning or
the end of the interval. WHAM is configured to have movement occur after
survival and mortality: \begin{equation*}
  \mathbf{O}_{y,a,i} = \mathbf{S}_{y,a,i}\boldsymbol{\mu}_{y,a,i}
\end{equation*} where \(\mathbf{S}_{y,a,i}\) is a \(n_R \times n_R\)
diagonal matrix of proportions surviving in each region (given they
start in that region) \begin{equation*}
\mathbf{S}_{y,a,i} = 
  \begin{bmatrix}
    e^{-Z_{y,a,i,1}\delta_i} & 0 & \cdots & 0 \\
    0 & e^{-Z_{y,a,i,2}\delta_i} & \cdots & 0 \\
    \vdots & \vdots & \ddots & \vdots \\
    0 & \cdots & 0 & e^{-Z_{y,a,i,n_R}\delta_i}
  \end{bmatrix}
\end{equation*} and \(\boldsymbol{\mu}_{y,a,i}\) is a \(n_R \times n_R\)
matrix of probabilities of moving from one region to another or staying
in the region they occurred at the beginning of the interval:
\begin{equation*}
\boldsymbol{\mu}_{y,a,i} = 
  \begin{bmatrix}
    1-\sum_{r' \neq 1} \mu_{1\rightarrow r',y,a,i} & \mu_{1\rightarrow 2,y,a,i} & \cdots & \mu_{1\rightarrow R,y,a,i} \\
    \mu_{2\rightarrow 1,y,a,i} & 1-\sum_{r' \neq 2} \mu_{2\rightarrow r',y,a,i} & \cdots & \mu_{2\rightarrow R,y,a,i} \\
    \vdots & \vdots & \ddots & \vdots \\
    \mu_{R\rightarrow 1,y,a,i} & \cdots & \mu_{R\rightarrow R-1,y,a,i} & 1-\sum_{r' \neq R} \mu_{R\rightarrow r',y,a,i}
  \end{bmatrix}
\end{equation*}

WHAM assumes each fleet \(f\) can harvest in only 1 region (\(r_f\))
during specified seasons. So, for each fleet \(f\), row \(r_f\) and
column \(f\) of \(\mathbf{H}_{y,a,i}\) will be
\(F_{y,a,i,f}\left(1 - e^{-Z_{y,a,i,r}\delta_i}\right)/Z_{y,a,i,r}\)
when fleet \(f\) is harvesting during the interval \(\delta_i\) and all
other elements will be zero. Row \(r\) of the single-column matrix
\(\mathbf{D}_{y,a,i}\) is
\(M_{y,a,r}\left(1 - e^{-Z_{y,a,i,r}\delta_i}\right)/Z_{y,a,i,r}\)

When survival and movement are assumed to occur simultaneously, all
movement and mortality parameters are instantaneous rates. We obtain the
probability transition matrix over an interval \(\delta_i\) by
exponentiating the instantaneous rate matrix (Miller and Andersen 2008)
\begin{equation*}
\mathbf{P}_{y,a,i} = e^{\mathbf{A}_{y,a,i}\delta_i}
\end{equation*} The instantaneous rate matrix takes rates of movement
between regions and the mortality rates for each fleet and region. Along
the diagonal is the negative of the sum of the other rates (the hazard)
so each row sums to zero. For two regions and one fleet operating in
each region: \begin{equation*}
 \mathbf{A}_{y,a,i} = \begin{bmatrix}
 a_{y,a,i,1} & \mu_{1\rightarrow 2,y,a,i} & F_{y,a,i,1} & 0 & M_{y,a,1} \\
 \mu_{2\rightarrow 1,y,a,i} &  a_{y,a,i,2} & 0 & F_{y,a,i,2} & M_{y,a,2} \\
 0 & 0 & 0 & 0 & 0 \\
 0 & 0 & 0 & 0 & 0 \\
 0 & 0 & 0 & 0 & 0
 \end{bmatrix}.
\end{equation*} where
\(a_{y,a,i,r} = -(\mu_{r\rightarrow r',y,a,i} + F_{y,a,i,r_f} + M_{y,a,r})\).
When there is one region, \(n_f\) fleets, and \(\delta_i = 1\),
exponentiating the instantaneous rate matrix results in the PTM defined
in Eq. \ref{eq:ptm_1_region}.

\hypertarget{seasonality}{%
\subsubsection*{Seasonality}\label{seasonality}}
\addcontentsline{toc}{subsubsection}{Seasonality}

Seasonality can be configured to accommodate characteristics of
spawning, movement, and fleet-specific behavior. The annual time step
can be divided into \(K\) (any number) seasons and the interval size
\(\delta_i\) for each season \(i\) need not be equal to any other
seasonal interval. Under the Markov assumption, the PTM of surviving and
moving and dying over \(K\) intervals \(\delta_1,\ldots, \delta_K\)
(i.e., the entire year) is just the product of the PTMs for each
interval:
\[ \mathbf{P}_{y,a}(\delta_1,\ldots,\delta_K) = \prod^K_{i=1}\mathbf{P}_{y,a,i}(\delta_i).\]

For a stock spawning at some fraction of the year \(0<t_s<1\) in
interval \(\delta_j\), the fraction of time in season \(j\) is
\[\delta_{s,j} = t_s-\sum^{j-1}_{i=0}\delta_i
\] and the PTM defining the proportions in each state at time \(t_s\)
for age \(a\) is \begin{equation}\label{eq:ptm_spawn}
\mathbf{P}_{y,a}\left(\delta_1,\ldots,\delta_{j-1}, \delta_{s,j}\right) = \mathbf{P}_{y,a}\left(t_s\right) =  \left[\prod^{j-1}_{i=1}\mathbf{P}_{y,a,i}(\delta_i)\right]\mathbf{P}_{y,a,j}(\delta_{s,j}).
\end{equation} Similarly, for an index \(m\) occurring at fraction of
the year \(t_m\) in interval \(\delta_j\) the proportions in each state
at the time of the observation is \begin{equation} \label{eq:ptm_index} 
\mathbf{P}_{y,a}\left(\delta_1,\ldots,\delta_{j-1}, \delta_{m,j}\right) = \mathbf{P}_{y,a}\left(t_m\right) =   \left[\prod^{j-1}_{i=1}\mathbf{P}_{y,a,i}(\delta_i)\right]\mathbf{P}_{y,a,j}(\delta_{m,j}).
\end{equation}

\hypertarget{numbers-at-age}{%
\subsubsection*{Numbers at age}\label{numbers-at-age}}
\addcontentsline{toc}{subsubsection}{Numbers at age}

When there are \(n_R\) regions and \(n_F\) fleets, The vector of
abundance in each state at age \(a>1\) on January 1 is
\(\mathbf{N}_{y,a} = (\mathbf{N}_{O,y,a}', \mathbf{0}')'\) where
\(\mathbf{N}_{O,y,a} = (N_{y,a,1}, \ldots, N_{y,a,n_R})'\) is the number
in the states corresponding to being alive in each region and
\(\mathbf{0}\) is a vector (\(n_F+1\)) for the numbers captured in each
fleet and dead from natural mortality because no age \(a\) fish have
died yet on January 1.

Each stock \(s\) is assumed to spawn and recruit in one region \(r_s\).
So for age \(a=1\), \(\mathbf{N}_{O,y,1}\) is 0 except for row
\(r = r_s\). Options for configuring recruitment (\(N_{y,1,r_s}\)) for
each stock are the same as previous versions of WHAM. If recruitment is
assumed to be a function of spawning stock biomass (SSB), it is only the
spawning population in region \(r_s\) at the time of spawning
constitutes the SSB in the stock-recruit function. However, models can
configure spawning individuals to occur in other regions at the time of
spawning. Aside from treating recruitment as a random walk, the general
model for annual recruitment as random effects is \begin{equation*}
\log\left(N_{y,1,r_s}\right)|\text{SSB}_{y-1,r_s} =  f\left(\text{SSB}_{y-1,r_s}\right) + \varepsilon_{y,1,r_s}
\end{equation*} where\\
\[\text{SSB}_{s,y} =  \sum^A_{a=1}  w_{s,y,a} \text{mat}_{s,y,a} \mathbf{O}_{s,y,a,r_s}(t_s)' \mathbf{N}_{O,s,y,a}\]
where \(w_{s,y,a}\) is the mean weight at age of spawning individuals,
\(\text{mat}_{s,y,a}\) is the maturity at age, and
\(\mathbf{O}_{s,y,a,r_s}(t_s)\), the \(r_s\) column of the upper-left
submatrix of Eq. \ref{eq:ptm_spawn}, are the probabilities of surviving
and occurring in region \(r_s\) at time \(t_s\) given being alive in
each region at the start of the year.

As in previous versions of WHAM, the transitions in numbers at age from
one year to another after recruitment can be treated deterministically
or as functions of random effects. The predicted numbers at age in year
\(y\) at age \(a\) for a given stock are vector analogs
(\(\mathbf{N}_{O,y,a}\)) of the equations for numbers at age in the
standard WHAM model (Stock and Miller 2021). For ages
\(a = 2,\ldots, A-1\), where \(A\) is the plus group, the expected
number alive in each region at the beginning of the following year and
next age class age can be obtained from the first \(n_R\) elements of
the vector \[\mathbf{P}_{y-1,a-1}' \mathbf{N}_{y-1,a-1}.\] The numbers
alive in each region can also be modeled more simply using the
sub-matrix \(\mathbf{O}_{y,a}\). The general model for the transitions
in abundance at age is \begin{equation*}
\log\left(\mathbf{N}_{O,y,a}\right)|\mathbf{N}_{O,y-1,a-1} =  \log\left(\mathbf{O}_{y-1,a-1}' \mathbf{N}_{O,y-1,a-1}\right) + \boldsymbol{\varepsilon}_{y,a}
\end{equation*} for ages \(a = 2,\ldots, A-1\), and for the plus group
\begin{equation*}
\log\left(\mathbf{N}_{O,y,A}\right)|\mathbf{N}_{O,y-1,a-1},\mathbf{N}_{O,y-1,A} = \log\left(\mathbf{O}_{y-1,A-1}' \mathbf{N}_{O,y-1,A-1} + \mathbf{O}_{y-1,A}' \mathbf{N}_{O,y-1,A}\right) + \boldsymbol{\varepsilon}_{y,A}.
\end{equation*} When the transitions in abundance at age are treated
deterministically, \(\boldsymbol{\varepsilon}_{y,a} = 0\). The stock-
and region-specific errors \(\boldsymbol{\varepsilon}_{y,a}\) are
independent, but the same correlation structures as previous versions
are possible across ages and years for a given stock and region. When
there is autocorrelation with age, WHAM now assumes this applies only to
ages \(a>1\) by default so that recruitment random effects are
independent of those for for the annual transitions of older age
classes. So the general covariance structure for a given stock at ages
\(a>1\) in region \(r\) is that of a two-dimensional first-order
autoregressive (2DAR1) process \begin{equation*}
  Cov\left(\epsilon_{y,a,r},\epsilon_{y',a',r}\right) =   \frac{\rho_{N,\text{age},r}^{|a-a'|}\rho_{N,\text{year},r}^{|y-y'|}\sigma_{N,a,r}\sigma_{N,a',r}}{\left(1 -  \rho_{N,\text{age},r}^2\right)\left(1 - \rho_{N,\text{year},r}^2\right)} 
\end{equation*} and that for age 1 is just AR1 across years
\begin{equation*}
  Cov\left(\epsilon_{y,1},\epsilon_{y',1}\right) =   \frac{\rho_{N,1,\text{year}}^{|y-y'|}\sigma^2_{N,1}}{1 - \rho_{N,1,\text{year}}^2}.
\end{equation*} Since recruitment for a given stock currently only
occurs in one region \(r_s\) there is only a single time-varying
recruitment random effect for each stock.

\hypertarget{initial-numbers-at-age}{%
\subsubsection*{Initial numbers at age}\label{initial-numbers-at-age}}
\addcontentsline{toc}{subsubsection}{Initial numbers at age}

Initial numbers at age for each stock and region can be treated as
age-specific fixed effects or with an equilibrium assumption as in
previous versions of WHAM. For the equilibrium option there are two
parameters for each stock: the stock-specific fully-selected fishing
mortality rate \(\log \widetilde{F}\) and the recruitment in year 1
\(\log N_{1,1,r_s}\). A stock-specific equilibrium fishing mortality at
age by fleet \({\tilde F}_{a,f}\) is the product of \(\widetilde{F}\)
and the selectivity across fleets in the first year \begin{equation*}
  \text{sel}_{1,a,f} = \frac{F_{1,a,f}}{\max_a \sum_{f=1}^{n_F} F_{1,a,f}}.
\end{equation*} We use \(\widetilde{F}_{a,f}\) to define an equilibrium
abundance per recruit by region at age \(a\) conditional on recruiting
to each region \begin{equation}\label{eq:npraa}
 \widetilde{\mathbf{O}}_{a} = \left\{
 \begin{array}{ll}
\prod^{a-1}_{j=0}\mathbf{O}_{j}  & 1\leq a<A\\
\left[\prod^{a-1}_{j=0}\mathbf{O}_{j}\right] \left(\mathbf{I} - \mathbf{O}_{A}\right)^{-1} & a = A
 \end{array}
\right.
\end{equation} where \(\mathbf{O}_{j}\) is the equilibrium probability
of surviving age \(a\) and occurring in each region and
\(\mathbf{O}_{0} = \mathbf{I}\). Natural mortality and movement rates in
the first year of the model are also used in Eq. \ref{eq:npraa}. For the
plus group \(a=A\), \(\left(\mathbf{I} - \mathbf{O}_{A}\right)^{-1}\) is
a ``fundamental matrix'\,' derived using the matrix version of the
geometric series (Kemeny and Snell 1960). Recall that recruitment for
stock \(s\) only occur in region \(r_s\) so, the equilibrium initial
numbers at age \(a\) by region are
\[\mathbf{N}_{O,1,a} = \widetilde{\mathbf{O}}_{a}' \mathbf{N}_{O,1,1}.\]

The initial abundances at age can also be treated as independent or as
AR1 random effects. Defining the vector of initial abundance at age in
region \(r\) as \(\mathbf{N}_{O,1,r}\), the general model is
\[\log \mathbf{N}_{O,1,r} = \theta_{N_1,r} + \boldsymbol{\varepsilon}_{N_1,r}\]
where
\[  Cov\left(\varepsilon_{N_1,a,r},\varepsilon_{N_1,a',r}\right) = \frac{\rho_{N_1,r}^{|a-a'|}\sigma^2_{N_1,r}} {\left(1-\rho_{N_1,r}^2\right)}.\]

\hypertarget{parametizing-movement}{%
\subsubsection*{Parametizing movement}\label{parametizing-movement}}
\addcontentsline{toc}{subsubsection}{Parametizing movement}

For each season, there are at most \(n_R-1\) parameters determining
movement among regions given starting an the season in region \(r\) in
either the sequential or simultaneous configurations. Movement
parameters are estimated on a transformed scale via a link function
\(g(\cdot)\). If survival and movement are occur simultaneously, the
parameters are estimated a log link function is used and if they are
separable, an additive logit link function (like a multinomial
regression) is used. On the transformed scale, the general model for the
movement parameter from region \(r\) to \(r'\) in season \(i\) and year
\(y\) for individuals of age \(a\) is a linear function of both random
and environmental effects: \begin{equation*}
  g(\mu_{r\rightarrow r',y,a,i}) = \theta_{r\rightarrow r',i} + \epsilon_{r\rightarrow r',y,a,i} + \sum^{n_E}_{k=1} \beta_{r \rightarrow r',a,i,k} E_{k,y}.
\end{equation*} The random effects \(\epsilon_{r\rightarrow r',y,a,i}\)
are season-, and(or) region-to-region-specific and modeled most
generally as 2DAR1 random effects with age and(or) year where the
covariance is \begin{equation*}
  Cov\left(\epsilon_{r\rightarrow r',y,a,i},\epsilon_{r\rightarrow r',y',a',i}\right) =   \frac{\rho_{r\rightarrow r',\text{age},i}^{|a-a'|}\rho_{r\rightarrow r',\text{year},i}^{|y-y'|}\sigma^2_{r\rightarrow r',i}}{\left(1 -  \rho_{r\rightarrow r',\text{age},i}^2\right)\left(1 - \rho_{r\rightarrow r',\text{year},i}^2\right)}
\end{equation*} similar to how WHAM models variation in survival,
natural mortality, and selectivity. Effects of covariate \(E_k\) can be
age-, season-, and(or) region-to-region-specific
\(\beta_{r\rightarrow r',a,i,k}\) and the same orthogonal polynomial
options in the previous versions of WHAM for effects on recruitment and
natural mortality are available.

There is currently no likelihood component for tagging data. Therefore,
movement parameters would generally either need to be fixed or assumed
to have some prior distribution, possibly based on external parameter
estimates. We include prior distributions for the season and
region-to-region specific (mean) movement parameters which are treated
as random effects with the mean defined by the initial value of the
fixed effect counterpart and standard deviation \begin{equation*}
  \gamma_{r\rightarrow r',i} \sim \text{N}\left(\theta_{r\rightarrow r',i}, \sigma^2_{r\rightarrow r',i}\right).
  \end{equation*} When priors are used, the movement is defined instead
as \begin{equation*}
  g(\mu_{r\rightarrow r',y,a,i}) = \gamma_{r\rightarrow r',i} + \epsilon_{r\rightarrow r',y,a,i} + \sum^{n_E}_{k=1} \beta_{r\rightarrow r',a,i,k} E_{k,y} 
  \end{equation*}

\hypertarget{natural-mortality}{%
\subsubsection*{Natural mortality}\label{natural-mortality}}
\addcontentsline{toc}{subsubsection}{Natural mortality}

Natural mortality options have been expanded in WHAM. When not
estimated, (mean) mortality rates may be stock-, region-, and
age-specific. When random effects are used, the same 2DAR1 structure
with age and year as described by Stock and Miller (2021) can be
configured for a given stock and region. Any environmental covariate
effects can be stock-, region-, and age-specific. So the general model
for natural mortality is \begin{equation*}
  \log M_{y,a,r} = \theta_{M,r} + \epsilon_{M,r,y,a} + \sum^{n_E}_{k=1} \beta_{M,r,a,k} E_{k,y}.
\end{equation*} The general covariance structure for random effects are
modeled most generally as 2DAR1 random effect with age and(or) year
where the covariance is \begin{equation*}
  Cov\left(\epsilon_{M,y,a,r},\epsilon_{M,y',a',r}\right) =   \frac{\rho_{M,\text{age},r}^{|a-a'|}\rho_{M,\text{year},r}^{|y-y'|}\sigma^2_{M,r}}{\left(1 -  \rho_{M,\text{age},r}^2\right)\left(1 - \rho_{M,\text{year},r}^2\right)}.
\end{equation*}

\hypertarget{catch-observations}{%
\subsubsection*{Catch observations}\label{catch-observations}}
\addcontentsline{toc}{subsubsection}{Catch observations}

The log-normal distributional assumption for aggregate catch
observations is the same as Stock and Miller (2021), but the predicted
catch is now a function of catch from each stock starting the year in
each region. For a given stock and age, the numbers captured in each
fleet over the year are
\[\widehat{\mathbf{N}}_{H,s,y,a} = \mathbf{H}_{s,y,a}' \mathbf{N}_{O,s,y,a}\]
The predicted numbers caught be each fleet across stocks is
\[\widehat{\mathbf{N}}_{H,y,a} = \sum^{n_S}_{s=1} \widehat{\mathbf{N}}_{H,s,y,a}\]
and the predicted aggregate catch at age \(a\) is
\[\widehat{\mathbf{C}}_{y,a} = \text{diag}\left(\mathbf{c}_{y,a}\right) \widehat{\mathbf{N}}_{H,y,a}\]
where \(\mathbf{c}_{y,a}\) is the vector of mean individual weight at
age \(a\) for each fleet and the aggregate catch by fleet is
\[\widehat{\mathbf{C}}_y = \sum^{A}_{a=1} \widehat{\mathbf{C}}_{y,a}\]
The log-aggregate catch observations for fleet \(f\) are normally
distributed
\[ \log C_{y,f} \sim \text{N}\left(\log \widehat {C}_{y,f}, \sigma^2_{y,f}\right).\]

The predicted numbers caught for each fleet \(f\) (row \(f\) of
\(\widehat{\mathbf{N}}_{H,y,a}\)) are used to make predicted age
composition observations as described by Stock and Miller (2021). Since
then, three additional likelihood options for age composition
observations have been added: a logistic-normal with AR(1) correlation
structure (Francis 2014), the alternative Dirichlet-multinomial
parameterization described by Thorson et al. (2017), and the
multivariate Tweedie (Thorson et al. 2023).

\hypertarget{index-observations}{%
\subsubsection*{Index observations}\label{index-observations}}
\addcontentsline{toc}{subsubsection}{Index observations}

For index \(m\) occurring in region \(r_m\) at fraction of the year
\(t_m\), the predicted abundance at \(t_m\) in region \(r_m\) is
\[\widehat{N}_{s,y,a,m} = \mathbf{O}_{s,y,a,r_m}(t_m)' \mathbf{N}_{O,s,y,a}\]
where \(\mathbf{O}_{s,y,a,r_m}(t_m)\), the \(r_m\) column of the
upper-left submatrix of Eq. \ref{eq:ptm_index}, are the probabilities of
surviving and occurring in region \(r_m\) at time \(t_m\) given being
alive in each region at the start of the year. The predicted index at
age is
\[\widehat{I}_{m,y,a} = q_{m,y} \text{sel}_{m,y,a}w_{m,y,a}\sum^{n_S}_{s = 1}\widehat{N}_{s,y,a,m}\]
where \(q_{m,y}\) is the catchability of the index in year \(y\),
\(\text{sel}_{m,y,a}\) is the selectivity and \(w_{m,y,a}\) is the
average weight of individuals at age \(a\) if the index is quantified in
biomass and \(w_{m,y,a} = 1\) if the index is quantified in numbers.
Predicted age composition observations are functions of
\(\widehat{I}_{m,y,a}\) as described by Stock and Miller (2021) and the
likelihood options are the same as those for catch explained above.

Catchability of the index can also be treated as functions of normal
random effects and(or) environmental covariate effects
\[log \frac{q_{m,y}-l_m}{u_m-q_{m,y}} = \theta_{q,m} + \varepsilon_{q,m,y}  + \sum^{n_E}_{k=1} \beta_{q,m,k} E_{k,y}\]
where \(u_{m}\) and \(l_{m}\) are the upper and lower bounds of
catchability for index \(m\) (defaults are 0 and 1000) and the general
covariance structure for the annual random effects is AR1
\[Cov\left(\epsilon_{q,m,y},\epsilon_{q,m,y'}\right) =   \frac{\rho_{q,m}^{|y-y'|}\sigma^2_{q,m}}{1 - \rho_{q,m}^2}.\]

\hypertarget{weight-and-maturity-at-age}{%
\subsubsection*{Weight and Maturity at
age}\label{weight-and-maturity-at-age}}
\addcontentsline{toc}{subsubsection}{Weight and Maturity at age}

Weight and maturity at age are treated similarly to Stock and Miller
(2021). Annual weight at age matrices for each fleet are used to
calculate total catch and the weight at age is applied to catch numbers
at age for any stocks caught by the fleet. Similarly, when indices
and(or) associated age composition observations are measured in biomass,
the weight at age matrices for the index are applied to predicted
numbers at age of all stocks observed by the survey. Unique weight at
age and maturity at age matrices are allowed for each stock to calculate
spawning stock biomass.

\hypertarget{reference-points}{%
\subsubsection*{Reference points}\label{reference-points}}
\addcontentsline{toc}{subsubsection}{Reference points}

Currently a single F reference point \(\widetilde F\) is estimated
across stocks and regions and F by fleet and age is
\(\widetilde F_{f,a} = \widetilde F \text{sel}_{f,a}\). Selectivity is
determined as before when there are multiple fleets where
\(\text{sel}_{f,a}\) is determined by averaging F at age over a
user-defined set of years \begin{equation*}
  \text{sel}_{a,f} = \frac{\overline F_{a,f}}{\max_a \sum^{n_F}_{f=1}{\overline F}_{a,f}}
\end{equation*}

The equilibrium spawning stock biomass per recruit for stock \(s\) in
region \(r_s\) is defined as \begin{equation}\label{eq:ssbpr}
 \upphi_s(\widetilde{F}) = \sum^{A}_{a=1} \widetilde{\mathbf{O}}_{s,a,r_s,\cdot} \mathbf{O}_{s,a,\cdot,r_s}(t_s) w_{s,a} m_{s,a}
\end{equation} where \(w_{s,a}\) and \(m_{s,a}\) are the mean individual
weight and probability of maturity at age \(a\),
\(\widetilde{\mathbf{O}}_{s,a}\) are as described in Eq. \ref{eq:npraa},
and \(\mathbf{O}_{s,a}(t_s)\) is the \(n_R \times n_R\) upper-left
sub-matrix of eq. \ref{eq:ptm_spawn} with the probabilities of surviving
and occurring in each region \(r'\) at age \(a+t_s\) given starting in
region \(r\) at age \(a\). The further subscripts \(r_s,\cdot\) and
\(\cdot,r_s\) indictate row or column \(r_s\), respectively. Using these
rows and columns is required because of the assumption that spawning and
recruitment only occur in region \(r_s\).

The equilibrium spawning biomass per recruit (eq. \ref{eq:ssbpr}) is
conditional on the region of recruitment \(r_s\). The equilibrium
recruitment in each region \(\widetilde {\mathbf{N}}_{s,1}\), depends on
the stock dynamics. This version of WHAM currently only allows complete
spawning region fidelity so that a stock only spawns and recruits in a
single region. In this case, \(\widetilde {\mathbf{N}}_{s,1}\) will be
positive in the spawning region (\(r_s\)) and zero elsewhere. Similarly,
the row \(r_s\) of \(\mathbf{O}_s\) will be zero off of the diagonal.
The matrices of probabilities of surviving and occurring in each region,
\(\widetilde{\mathbf{O}}_{s,a}\) and \(\mathbf{O}_{s,a}(\delta_s)\), are
functions of the fishing mortality rates for fleets in each region
\(\widetilde F_{f,a}\).

The matrix equilibrium yield per recruit as a function of
\(\widetilde F\) is calculated as \begin{equation}\label{eq:ypr}
 \widetilde{Y}_s({\widetilde{F}}) = \sum^{A}_{a=1} \widetilde{\mathbf{O}}_{s,a,r_s,\cdot} \mathbf{H}_{s,a} \mathbf{c}_{s,a}
\end{equation} where \(\mathbf{c}_a\) is the vector of mean individual
weight at age for each fleet, and \(\mathbf{H}_{s,a}\) is the submatrix
of the probabilities of being captured in each fleet over the interval
from \(a\) to \(a+1\), defined in eq. \ref{eq:ptm}.

As in previous versions of WHAM package, ``static'\,' reference points,
typically meant to be defined for prevailing conditions, average all of
the inputs to the spawning biomass and yield per recruit calculations
over the user-specified years (e.g., last 5 years of the model). This
same averaging is also applied to possibly time-varying movement
parameters.

For \(X\%\) SPR-based reference points, we use a Newton method and
iterate \begin{equation}\label{eq:newton}
  \log\widetilde{F}^{(i)} = \log\widetilde{F}^{(i-1)} - \frac{g\left(\log\widetilde{F}^{(i-1)}\right)}{g'\left(\log\widetilde{F}^{(i-1)}\right)}
\end{equation} where \(g(\log F)\) is the difference between the
weighted sums of spawning biomass per recruit at \(F\) and \(X\)\% of
unfished spawning biomass per recruit across stocks: \begin{equation*}
  g(\log F) = \sum^{n_s}_{s=1} \lambda_s\left[\upphi_s\left(F = e^{\log F}\right) - \frac{X}{100}\upphi_s\left(F=0\right)\right].
\end{equation*} where \(\upphi_s\left(F=0\right)\) is the equilibrium
unfished spawning biomass per recruit. \(g'(\log F)\) is the derivative
of \(g\) with respect to \(\log F\), and the weights to use for each
stock \(\lambda_s\) can be specified by the user or relative to the
average of recruitment for each stock over the same years the user
defines to calculate ``static'\,' equilibrium spawning biomass and
yield.

When a Beverton-Holt or Ricker stock recruit relationship is assumed, an
analogous Newton method is used to find \(\log F\) that maximizes yield
for MSY-based reference points. which are also a functions of the
equilibrium yield per recruit (\ref{eq:ypr}) and equilibrium
recruitment. The function \(g(\log F)\) in Eq. \ref{eq:newton} is the
first derivative of the yield curve with respect to \(\log F\).

\hypertarget{projections}{%
\subsubsection*{Projections}\label{projections}}
\addcontentsline{toc}{subsubsection}{Projections}

The projection options are generally the same as those for previous
versions of WHAM. When there is movement of any stocks, the user has the
option to project and use any random effects for time-varying movement
or use the average over user specified years, analogous to how natural
mortality can be treated in the projection period. The projection of any
environmental covariates has been revised to better include error in the
estimated latent covariate in any effects on the population in
projection years.

\hypertarget{application-to-black-sea-bass}{%
\section*{Application to black sea
bass}\label{application-to-black-sea-bass}}
\addcontentsline{toc}{section}{Application to black sea bass}

Prior to its 2023 peer-reviewed assessment, the NEUS black sea bass
stock was assessed using the Age-Structured Assessment Program (ASAP)
model (Legault and Restrepo 1999), a single-stock and -region
statistical catch-at-age model that estimates all model parameters as
fixed effects. Northern and southern components of the NEUS black sea
bass stock ascribed to regions divided by the Hudson Canyon were
separately modeled in ASAP (reference map figure). Results from the
separate ASAP models were combined for a unit-stock assessment. The
ASAP-based assessments exhibited strong retrospective patterns (Mohn
1999), and exploring alternative modeling approaches for the northern
and southern stock components has been a high priority for management.

Leading up to the 2023 peer-reviewed assessment, a working group
(hereafter referred to as ``working group'') composed of scientists from
federal, state, and academic institutions determined an optimal data and
model configuration for the black sea bass stock using the multi-stock
and multi-region extension of WHAM described above (NEFSC 2023). This
assessment included the spatial features and investigated inclusion of
hypothesized environmental drivers that were prioritized research
recommendations from previous black sea bass assessments.

Below we describe the assumptions and configuration of the assessment
model as determined by the working group as well as the alternative
assumptions for recruitment and natural mortality in the models we fit
to evaluate alternative hypotheses of bottom temperature effects on
black sea bass.

\hypertarget{basic-structure}{%
\subsection*{Basic structure}\label{basic-structure}}
\addcontentsline{toc}{subsection}{Basic structure}

The first year being modeled for the population is 1989 and the fishery
and index data used in the model span from 1989 to 2021. The north and
south stock components are modeled as separate populations that spawn
and recruit in respective regions. We have observations for each of four
total fishing fleets, where two fishing fleets (Recreational and
Commercial) operate in each region.

There are 11 seasonal intervals within each calendar year: five monthly
time intervals from Jan 1 to May 31, a spawning season from June 1 to
July 31, and five monthly intervals from August 1 to December 31. The
southern stock component is assumed to never move to the northern
region. For the northern component, a proportion
\(\mu_{\text{N}\rightarrow \text{S}}\) can move to the southern region
each month during the last five months of the year, but no movement is
allows from the south to the north during this period (Figure
\ref{fig:migration-diagram}). During the first four intervals of the
year a proportion \(\mu_{\text{S}\rightarrow \text{N}}\) the northern
component individuals in the south can move back to the north, but no
movement from the north to south is allowed during this period. In the
fifth interval (May), all northern component individuals remaining in
the south are assumed to move back to the north for the subsequent
spawning period. Survival and movement occur sequentially in each
interval and each of the two movement proportions are assumed constant
across intervals, ages, and years.

The two monthly movement matrices are \begin{equation*}
\boldsymbol{\mu}_{1} = 
  \begin{bmatrix}
     1-\mu_{\text{N}\rightarrow \text{S}} & \mu_{\text{N}\rightarrow \text{S}} \\
     0 & 1 \\
  \end{bmatrix}
\end{equation*} for the portion of the year after spawning and
\begin{equation*}
\boldsymbol{\mu}_{2} = 
  \begin{bmatrix}
     1 &  0 \\
     \mu_{\text{S}\rightarrow \text{N}} & 1-\mu_{\text{S}\rightarrow \text{N}} \\
  \end{bmatrix}
\end{equation*} for the portion of the year before spawning. As noted in
the description of the general WHAM model, tagging data are not yet
allowed. However, the working group also fit a Stock Synthesis model
(Methot and Wetzel 2013) which provided estimated movement rate
parameters and standard errors that were used to configure the priors
for WHAM (see Supplementary Materials).

With the movement configuration, the northern origin fish (ages 2+) can
occur in the southern region on January 1. Estimating initial numbers at
age as separate parameters can be challenging even in single-stock
models, but for black sea bass the available data cannot distinguish the
proportion of northern and southern component fish at each age in the
southern region in the initial year of the model. Therefore, we used the
simplifying equilibrium assumption described above where there are two
parameters estimated for each stock component: an initial recruitment
and an equilibrium fully-selected \(F\) that determines the abundance at
age in each region for each stock component.

For the northern population abundance at age 1 on January 1
(recruitment) is only allowed in the northern region, but given the
monthly movement described above, older individuals that previously
recruited in the northern region may occur in the southern region on
January 1. Therefore, a model with survival random effects will model
the transitions (survival/movement) of abundances at age of northern
origin fish in each region. All of the initial runs assumed variance
parameters for these random effects to be the same for northern origin
fish occurring in both regions on January 1. The base model assumes very
small variance for the transitions of northern fish in the southern
region, which is approximately the same as the deterministic transition
assumptions of a statistical catch at age model. We also allow 2DAR1
correlation for recruitment and survival for both the northern and
southern components. Unique variance and correlation parameters for the
recruitment and survival random effects are estimated for the northern
and southern components.

\hypertarget{uncertainty-recreational-cpa-index-observations}{%
\subsection*{Uncertainty recreational CPA index
observations}\label{uncertainty-recreational-cpa-index-observations}}
\addcontentsline{toc}{subsection}{Uncertainty recreational CPA index
observations}

The estimated coefficients of variation (CVs) provided by the analyses
to generate the recreational catch per angler (recreational CPA) index
ranged between 0.02 and 0.06 which the working group felt did not
capture the true observation uncertainty in the index with regard to its
relationship to stock abundance. In many of the initial runs as well as
the base model we allowed a scalar multiple of the standard deviation of
the log aggregate index to be estimated for these indices in the
northern and southern regions. Models that successfully estimated these
scalars indicated standard deviations for these surveys to be
approximately 5 times the input value and this value was fixed in many
preliminary runs to avoid dealing with convergence problems. However,
the base model successfully estimated these scalars. The model estimates
are negligibly affected by estimating these scalars, but we felt
estimating these parameters allowed uncertainty in model output to be
more properly conveyed.

\hypertarget{index-and-catch-age-composition-observations}{%
\subsection*{Index and Catch age composition
observations}\label{index-and-catch-age-composition-observations}}
\addcontentsline{toc}{subsection}{Index and Catch age composition
observations}

The working group investigated many alternative assumptions for the
probability models and selectivity models for the 8 different sets of
age composition observations to reduce residual patterns and
retrospective patterns. These analyses resulted in use of selectivity
random effects for the northern fleet and indices and logistic-normal
likelihoods for 6 sets of of age composition observations and
Dirichlet-multinomial likelihoods for one index and one fleet in the
northern region (Table \ref{tab:age-comp-sel-table}).

\hypertarget{bottom-temperature-effects}{%
\subsection*{Bottom Temperature
effects}\label{bottom-temperature-effects}}
\addcontentsline{toc}{subsection}{Bottom Temperature effects}

All model fits also include bottom temperature observations for the
northern and southern regions from 1963 to 2021 and estimated standard
errors ranging between 0.03 and 0.09 degrees Celsius (NEFSC 2023). We
retained the assumption from the peer-reviewed assessment that treated
the latent bottom temperature covariates in each region as AR1
processes.

We fit 14 models with alternative assumptions about the effects of
bottom temperature covariates, ranging from no effects to effects on
both regions for either recruitment or natural mortality at age 1 (Table
\ref{tab:model-desc-table}). These analyses derive from the hypothesis
that bottom temperature affects overwinter survival of fish where the
fish turn from age 0 to age 1 on January 1 (Miller et al. 2016a). This
temperature may be a proxy for temperature prior to January 1 and affect
survival during the end of the pre-recruit phase or natural mortality in
the early part of of the year after becoming age 1. Furthermore, we have
no direct observations of age 1 individuals from surveys until the
spring season each year. Therefore, we fit models with effects of
temperature on recruitment or natural mortality at age 1.

It is standard practice to treat annual recruitment as time-varying
deviations from a mean model and all models here treat recruitment
deviations as AR1 random effects. However, Miller et al. (2018) showed
how inferences of temperature effects on growth or maturity parameters
can be very different whether the compared models with and without the
effect also include random effects representing residual temporal
variation in parameters. Therefore, we also explored whether including
temporal random effects on age 1 natural mortality affected inferences
on corresponding temperature effects. Initially, we included random
effects on age 1 natural mortality for both the northern and southern
stock components, but estimates of these random effects and
corresponding variance for the southern component converged to 0 so
these were not included in models presented here.

We assume the covariate in year \(y\) affects recruitment in the same
years because the covariate observations are from months January to
March. The fish are technically already 1 year old, but there are no
observations of these individuals until later in the year except
possibly in fishery catches which are accumulated over the whole year.
Expected log-recruitment for a given stock is a linear function of
bottom temperature

\begin{equation}\label{eq:expected-recruitment}
E\left(\log N_{y,1}|x_y\right) = \mu_{R} + \beta_{R} x_y.
\end{equation} Similarly, expected log-natural morality as a function of
bottom temperature is \begin{equation}\label{eq:expected-M1}
E\left(\log M_{y,1}|x_y\right) = \mu_{M,1} + \beta_{M} x_y
\end{equation} Because age 1 fish for the northern component can exist
in both regions after January 1, natural mortality is acting in both
regions for this stock. For models with covariate and/or annual random
effects for age 1 fish we assume them only in the northern regions for
the northern component. The corresponding random effects are
\begin{equation}\label{eq:Rec-re}
\log N_{y,1} = E\left(\log N_{y,1}|x_y\right) + \epsilon_{y,1}
\end{equation} and \begin{equation}\label{eq:M-re}
\log M_{y,1} = E\left(\log M_{y,1}|x_y\right) + \epsilon_{M,y}.
\end{equation} The constant or mean log-natural mortality rate is
assumed to be \(\mu_{M,a} = \log(0.4)\) for all ages as recommended by
the working group. Because the bottom temperature anomalies and random
effects are centered at 0, mean log-natural mortality at age 1 over the
time series should be approximately equal to \(\mu_{M,1}\) for models
that include those effects.

\hypertarget{model-fitting-diagnostics-and-projections}{%
\subsection*{Model fitting, diagnostics, and
projections}\label{model-fitting-diagnostics-and-projections}}
\addcontentsline{toc}{subsection}{Model fitting, diagnostics, and
projections}

We used a development version
(\href{https://github.com/timjmiller/wham/tree/fb8b089}{commit fb8b089})
of the WHAM package prior to the release of version 2.0 for all results.
All code for fitting models and generating results can be found at
\href{https://github.com/timjmiller/wham_bsb_paper}{github.com/timjmiller/wham\_bsb\_paper}.

We examined retrospective patterns for all models by fitting
corresponding models where the terminal year is reduced sequentially by
one year (peel) for seven years. Therefore, there are eight fits of each
model with the time series reduced by zero to seven years. We calculated
Mohn's \(\rho\) for SSB, and average \(F\) at ages 6 and 7. Absolute
values of Mohn's \(\rho\) near 0 imply no pattern in estimation of these
quantities as the time-series is sequentially extended.

As in Miller et al. (2016b), we also assessed the consistency of the
AIC-based model selection over retrospective peels to guard against
previously noted changes in perception of covariate effects on
recruitment with increased length of the time series of observations
(Myers 1998). This retrospective examination was also recommended by
Brooks (2024).

We performed a jitter analysis of the base model \(M_0\) and the best
fitting model to investigate whether a local mimimum of the negative
log-likelihood surface was obtained by the optimization. We used the
jitter\_wham function in the WHAM package which by default simulates
starting values from a multivariate normal distribution with mean and
covariance defined by the MLEs and estimated covariance matrix from the
fitted model. See the Supplementary Materials for more details.

For the best performing model, we also performed so-called simulation
self-tests where new observations were simulated conditional on all
estimated fixed and random effects and the same model configuration was
fit to each of the simulated data sets. We estimated median relative
bias of SSB across these simulations. See the Supplementary Materials
for more details.

To illustrate projections with environmental effects we projected the
best black sea bass model under three alternative scenarios where the
AR1 time-series model for bottom temperature continues into the
projection years, where we project with bottom temperature being the
average of the most recent 5 years, and where bottom temperature
increases following a predictions from a simple linear regression of the
estimated bottom temperature anomalies over time.

\hypertarget{results}{%
\section*{Results}\label{results}}
\addcontentsline{toc}{section}{Results}

We found the best model of bottom temperature covariate effects included
the effect only on the recruitment of the northern stock (Table
\ref{tab:diff-aic-table}). However, the difference in AIC for the model
that also included effects on recruitment for the southern component
suggested some evidence for this hypothesis as well (Table
\ref{tab:aic-wts-table}).

Across retrospective peels where the terminal year of data was
sequentially reduced, the ranking for the best model was consistent.
However, when the terminal year was reduced 6 or more years, there was
more evidence for further effects of temperature on recruitment for the
southern component and for temporal variation in natural mortality at
age 1 (Table \ref{tab:aic-wts-table}).

The retrospective model fits did not indicate any evidence of patterns
for northern or southern component SSB (all Mohn's
\(\rho \approx -0.03\)) or fishing mortality (all Mohn's
\(\rho \approx 0.03\) and \(-0.04\) for average \(F\) at ages 6 and 7 in
north and south regions, respectively) for any of the models (Table
\ref{tab:rho-table}).

\hypertarget{bottom-temperature-effects-1}{%
\subsection*{Bottom temperature
effects}\label{bottom-temperature-effects-1}}
\addcontentsline{toc}{subsection}{Bottom temperature effects}

The posterior estimate of the bottom temperature covariate match the
observations well because of the high precision of the observations and
anomalies in the north and south region appear highly correlated (Figure
\ref{fig:bottom-temperature}). Because the temperature anomalies are
treated as latent variables, when effects on recruitment are included,
other data components in the model can affect the estimated anomalies
(e.g., Miller et al. 2018), but in this case the estimates are altered
negligibly during the years where recruitment is estimated (Figure
\ref{fig:Ecov-M1-rel-M0}).

Estimated effects of bottom temperature on recruitment for either the
northern or southern component were stable over the retrospective peels
and differed negligibly whether effects for each component were
estimated in isolation or together (Table
\ref{tab:beta-sig-peel-table}). Estimates of residual variability in
northern component recruitment, as measured by the conditional or
marginal standard deviation of the recruitment random effects, increased
slightly with the number of peeled years. However, the ratio of standard
deviations of models with and without temperature effects was stable
with about a 20\% reduction in residual standard deviation when
temperature effects were included.

Although evidence for temperature effects on age 1 natural mortality was
weak, it is notable that the sign of the estimated effect for the
northern stock component differed depending on whether random effects on
age 1 natural mortality were also included (\(\widehat \beta_R = -1\)
for \(M_{6}\) or \(M_{13}\)) with the negative effect estimated with
these random effects (\(M_{13}\)).

\hypertarget{stock-size-fishing}{%
\subsection*{Stock size, fishing}\label{stock-size-fishing}}
\addcontentsline{toc}{subsection}{Stock size, fishing}

For the best performing model \(M_1\), estimates of SSB Stock size is
also increasing for the southern component, just not as quickly as the
northern component. (plot of SSB time series for each component).

Figure \ref{fig:M1-SSB-F-R}

Figure \ref{fig:SSB-F-R-rel-M1} Annual estimates of SSB and F varied
little among the 14 fitted models.

plot of prior/posterior for movement rate.

plot of average F by fleet,region.

Sectivity random effects (Figure \ref{fig:selectivity-re})

Estimates of effect size, sd, correlation for each peel (Table \ref{}

The residual variation in the standard deviation of recruitment random
effects is reduced because the expected recruitment (Eq.
\ref{eq:expected-recruitment}) is a function of the covariate. This
effect is included in the proposed base model (Figure
\ref{fig:BT-Ecov-R}).

plot of northern component predicted recruitment vs.~time with color by
bottom temperature

\hypertarget{reference-points-1}{%
\subsection*{Reference points}\label{reference-points-1}}
\addcontentsline{toc}{subsection}{Reference points}

\hypertarget{projections-1}{%
\subsection*{Projections}\label{projections-1}}
\addcontentsline{toc}{subsection}{Projections}

Projections of bottom temperature and recruitment and CVs (Figure
\ref{fig:R-BT-proj})

Uncertainty in both is much lower during the data years than the
projection years.

Uncertainty in BT RE propagates into both expected recruitment and
estimated random effect for recruitment, but the latter also includes
uncertainty due to the residual variability in recruitment.

\hypertarget{discussion}{%
\section*{Discussion}\label{discussion}}
\addcontentsline{toc}{section}{Discussion}

\hypertarget{general-model-aspects}{%
\subsection*{General Model aspects}\label{general-model-aspects}}
\addcontentsline{toc}{subsection}{General Model aspects}

WHAM provides a comprehensive treatment of environmental covariates and
their effects on populations using state-space methods (Maunder 2024).
This framework accounts for the magnitude of differing uncertainties in
observations and stochastic population dynamics processes. This version
2.0 extends these inferences to movement rates for multi-region models.

\hypertarget{future-developments}{%
\subsection*{Future developments}\label{future-developments}}
\addcontentsline{toc}{subsection}{Future developments}

An obvious limitation to WHAM is the inability to include tagging
observations of any type. Such observations are critical to estimation
of movement parameters (e.g., Goethel et al. 2019), but also can inform
mortality rate parameters (Hampton 1991). It is well known that natural
mortality is seldom estimated in assessment models because the observed
data often provide little information to distinguish natural mortality
from other assessment model parameters (Lee et al. 2011; Clark 2022).
Estimation of natural mortality may be even more challenging within
state-space assessment models with their greater flexibility from
inclusion of time-varying random effects. For example, we fixed the mean
natural mortality for age 1 fish and Cadigan (2016) and Stock et al.
(2021) also estimated natural mortality deviations. Therefore allowing
tagging observations in WHAM should be a high priority even for models
with a single stock and region.

The work by Correa et al. (2023) on incorporating length information and
modeling growth within WHAM has not yet been merged into the multi-stock
version 2.0. Growth and movement would be challenging.

\hypertarget{black-sea-bass}{%
\subsection*{Black sea bass}\label{black-sea-bass}}
\addcontentsline{toc}{subsection}{Black sea bass}

In our investigation of bottom temperature effects on recruitment and
natural mortality and time-varying random effects on natural mortality
at age 1, we found only including effects on recruitment for the
northern stock component to be the best model with respect to AIC. This
is the same configuration configuration accepted during the assessment
peer-review process and it is the model currently used for management
(NEFSC 2023).

Although not relevant to the best performing model, the effect of bottom
temperature on age 1 M would be expected to be opposite of that for
recruitment because lower M with higher temperature would produce higher
recruitment. This expected effect was estimated only when random effects
on age 1 M were also included in the model. This seems to provide
further support to including random effects when examining covariate
effects on population parameters as demonstrated by Miller et al. (2018)
for effects on growth parameters.

AIC suggested some weight for models with temperature effects on
recruitment for the southern component and random effects on M at age 1.
This might suggest making inferences on the population by using an
ensemble of these and weighting by AIC (Burnham and Anderson 2002).
However, the estimates of assessment outputs relevant for management
(SSB, F) were similar among the models and weighted estimates would
differ very little from the estimates with the lowest AIC. {[}but what
about reference points?{]}

\hypertarget{temperature-effects}{%
\subsubsection*{Temperature effects}\label{temperature-effects}}
\addcontentsline{toc}{subsubsection}{Temperature effects}

BSB in the NEUS is at the northern extent of its range. hence the range
extension to the north with increased temperature. The evidence for
temperature effects on recruitment was strong only for the northern
component which is the component of the stock at the limit of the
species' range. Opposite effects of temperature on recruitment or
population size would be expected for species in the same general area
that are at the southern extent of their range (Gabriel 1992). For
example, higher recruitment of the Southern New England-Mid-Atlantic
stock of yellowtail flounder is correlated with more cold water
persistence into summer fall on the Northeast US shelf Miller et al.
(2016b).

\hypertarget{self-tests}{%
\subsubsection*{Self Tests}\label{self-tests}}
\addcontentsline{toc}{subsubsection}{Self Tests}

Simulation self tests performed by Stock and Miller (2021) simulated
process errors and observation errors and their simulations showed
little bias in estimated assessment output such as SSB. Li et al. (2024)
used the same type of simulations and found what?.

Here we conditioned on the estimated process errors for black sea bass.
This conditional approach has been suggested as appropriate for tactical
management where we are interested in inferences assuming our estimation
of the population and fishing history from the original data is the
truth Cadigan et al. (2024). We found some bias in SSB estimation for
the black sea bass application (negatively for the northern stock),
using the default model. We also found instability in the estimation of
the observation variance parameters for the aggregate indices such that
they often were estimated at the lower bound of 0. However, we found
that the magnitude of bias was largely influenced by the large
uncertainty estimated in some of the age composition observations
assumed to have a logistic-normal distribution. When using the
logistic-normal likelihood for management, we recommend inspecting the
size of the estimated (marginal) standard deviation of the logistic
normal model when self-test bias is non-negligible. We also recommend
investigating the multiplicative transformation for the logistic normal
rather than the additive transformation which is currently the only
option in WHAM (Cadigan 2016).

?Selectivity RE are only on index age comp in the north which may be why
bias is larger there?

? We found REML provided less biased estimation of SSB and recruitment
which are functions of random effects. Whether REML should be the
default approach for estimating these quantities from state-space stock
assessment models requires further investigation. For example,
improvement in bias may be negligible when data contain low observation
error and/or there are few fixed effects parameters. ? larger variance
estimates allows greater spread of random effects estimates. Important
for conditional self-test where estimated RE are of primary importance
(Recruitment and SSB). When simulating RE and refitting, the sign of
differences between estimates and true values can be random across
simulations. Probably more important with more fixed effects and less
informative data

Including tagging data in WHAM would also be helpful for BSB because we
are using the same data to estimate movement outside of the model as we
are using to estimate the other parameters in WHAM.

\hypertarget{conclusions}{%
\subsection*{Conclusions}\label{conclusions}}
\addcontentsline{toc}{subsection}{Conclusions}

WHAM version 2.0 extends the existing R package to allow multiple stocks
and discrete spatial regions. It allows autocorrelated random effects
and environmental covariate effects on movement rates. Estimates of
movement rates from auxiliary studies can be used to construct prior
distributions for movement rates in lieu of integrated likelihoods for
tagging data. Version 2.0 also allows seasonal treatment of different
fishing fleets and movement of individual stocks, and internal
estimation of SPR- and MSY-based reference points accounting for
seasonal spatial dynamics including movement and mortality and any
environmental covariate effects.

We applied this extended version to investigate alternative hypotheses
about effects of bottom temperature on recruitment and age 1 natural
mortality for black sea bass. Our analyses indicate evidence for effects
of bottom temperature on recruitment for the northern component
represents was stronger than other models that included effects on age 1
mortality and/or corresponding effects on the southern stock component.

Future WHAM development should prioritize including tagging data to
inform stock assessment parameter estimation and merging in the version
created by Correa et al. (2023) to allow modeling of growth and
inclusion of length and length-at-age composition observations.

\hypertarget{references}{%
\section*{References}\label{references}}
\addcontentsline{toc}{section}{References}

\hypertarget{refs}{}
\begin{CSLReferences}{1}{0}
\leavevmode\vadjust pre{\hypertarget{ref-andersenkeiding02}{}}%
Andersen, P.K., and Keiding, N. 2002. Multi-state models for event
history analysis. Statistical Methods in Medical Research \textbf{11}:
91--115.

\leavevmode\vadjust pre{\hypertarget{ref-arnason72}{}}%
Arnason, A.N. 1972. Parameter estimates from mark-recapture experiments
on two populations subject to migration and death. Researches on
Population Ecology \textbf{13}: 97--113.
doi:\href{https://doi.org/10.1007/BF02521971}{10.1007/BF02521971}.

\leavevmode\vadjust pre{\hypertarget{ref-bosleyetal22}{}}%
Bosley, K.M., Schueller, A.M., Goethel, D.R., Hanselman, D.H., Fenske,
K.H., Berger, A.M., Deroba, J.J., and Langseth, B.J. 2022. Finding the
perfect mismatch: Evaluating misspecification of population structure
within spatially explicit integrated population models. Fish and
Fisheries \textbf{23}(2): 294--315.
doi:\href{https://doi.org/10.1111/faf.12616}{10.1111/faf.12616}.

\leavevmode\vadjust pre{\hypertarget{ref-brooks24}{}}%
Brooks, E.N. 2024. Pragmatic approaches to modeling recruitment in
fisheries stock assessment: A perspective. Fisheries Research
\textbf{270}: 106896.
doi:\href{https://doi.org/10.1016/j.fishres.2023.106896}{10.1016/j.fishres.2023.106896}.

\leavevmode\vadjust pre{\hypertarget{ref-burnhamanderson02}{}}%
Burnham, K.P., and Anderson, D.R. 2002. Model selection and multimodel
inference: A practical information-theoretic approach. Springer-Verlag,
New York.

\leavevmode\vadjust pre{\hypertarget{ref-cadigan16}{}}%
Cadigan, N.G. 2016. A state-space stock assessment model for northern
cod, including under-reported catches and variable natural mortality
rates. Canadian Journal of Fisheries and Aquatic Sciences
\textbf{73}(2): 296--308.
doi:\href{https://doi.org/10.1139/cjfas-2015-0047}{10.1139/cjfas-2015-0047}.

\leavevmode\vadjust pre{\hypertarget{ref-cadiganetal24}{}}%
Cadigan, N.G., Albertsen, C.M., Zheng, N., and Nielsen, A. 2024. Are
state-space stock assessment model confidence intervals accurate? Case
studies with SAM and barents sea stocks. Fisheries Research
\textbf{272}: 106950.
doi:\href{https://doi.org/10.1016/j.fishres.2024.106950}{10.1016/j.fishres.2024.106950}.

\leavevmode\vadjust pre{\hypertarget{ref-caoetal14}{}}%
Cao, J., Truesdell, S.B., and Chen, Y. 2014. Impacts of seasonal stock
mixing on the assessment of {A}tlantic cod in the {G}ulf of {M}aine.
ICES Journal of Marine Science \textbf{71}(6): 1443--1457.
doi:\href{https://doi.org/10.1093/icesjms/fsu066}{10.1093/icesjms/fsu066}.

\leavevmode\vadjust pre{\hypertarget{ref-clark22}{}}%
Clark, W.G. 2022. Why natural mortality is estimable, in theory if not
in practice, in a data-rich stock assessment. Fisheries Research
\textbf{248}: 106203.
doi:\href{https://doi.org/10.1016/j.fishres.2021.106203}{10.1016/j.fishres.2021.106203}.

\leavevmode\vadjust pre{\hypertarget{ref-commenges99}{}}%
Commenges, D. 1999. Multi-state models in epidemiology. Lifetime Data
Analysis \textbf{5}: 315--327.

\leavevmode\vadjust pre{\hypertarget{ref-correaetal23}{}}%
Correa, G.M., Monnahan, C.C., Sullivan, J.Y., Thorson, J.T., and Punt,
A.E. 2023. Modelling time-varying growth in state-space stock
assessments. ICES Journal of Marine Science \textbf{80}(7): 2036--2049.
doi:\href{https://doi.org/10.1093/icesjms/fsad133}{10.1093/icesjms/fsad133}.

\leavevmode\vadjust pre{\hypertarget{ref-francis14}{}}%
Francis, R.I.C.C. 2014. Replacing the multinomial in stock assessment
models: A first step. Fisheries Research \textbf{151}: 70--84.
doi:\href{https://doi.org/10.1016/j.fishres.2013.12.015}{10.1016/j.fishres.2013.12.015}.

\leavevmode\vadjust pre{\hypertarget{ref-gabriel92}{}}%
Gabriel, W.L. 1992. Persistence of demersal fish assemblages between
{C}ape {H}atteras and {N}ova {S}cotia, {N}orthwest {A}tlantic. Journal
of the Northwest Atlantic Fisheries Science \textbf{14}(1): 29--46.

\leavevmode\vadjust pre{\hypertarget{ref-goetheletal19}{}}%
Goethel, D.R., Bosley, K.M., Hanselman, D.H., Berger, A.M., Deroba,
J.J., Langseth, B.J., and Schueller, A.M. 2019. Exploring the utility of
different tag-recovery experimental designs for use in spatially
explicit, tag-integrated stock assessment models. Fisheries Research
\textbf{219}: 105320.
doi:\href{https://doi.org/10.1016/j.fishres.2019.105320}{10.1016/j.fishres.2019.105320}.

\leavevmode\vadjust pre{\hypertarget{ref-hampton91}{}}%
Hampton, J. 1991. Estimation of southern bluefin tuna \emph{{T}hunnus
maccoyii} natural mortality and movement rates from tagging experiments.
Fishery Bulletin \textbf{89}(4): 591--610.

\leavevmode\vadjust pre{\hypertarget{ref-hearnetal87}{}}%
Hearn, W.S., Sundland, R.L., and Hampton, J. 1987. Robust estimation of
the natural mortality rate in a completed tagging experiment with
variable fishing intensity. Journal Du Conseil International Pour
L'exploration De La Mer \textbf{43}: 107--117.
doi:\href{https://doi.org/10.1093/icesjms/43.2.107}{10.1093/icesjms/43.2.107}.

\leavevmode\vadjust pre{\hypertarget{ref-kemenysnell60}{}}%
Kemeny, J.G., and Snell, J.L. 1960. Finite markov chains. D. Van
Nostrand Company, Princeton, New Jersey.

\leavevmode\vadjust pre{\hypertarget{ref-kristensenetal16}{}}%
Kristensen, K., Nielsen, A., Berg, C., Skaug, H., and Bell, B.M. 2016.
{TMB}: Automatic differentiation and {Laplace} approximation. Journal of
Statistical Software \textbf{70}: 1--21.
doi:\href{https://doi.org/10.18637/jss.v070.i05}{10.18637/jss.v070.i05}.

\leavevmode\vadjust pre{\hypertarget{ref-leeetal11}{}}%
Lee, H.-H., Maunder, M.N., Piner, K.R., and Methot, R.D. 2011.
Estimating natural mortality within a fisheries stock assessment model:
An evaluation using simulation analysis based on twelve stock
assessments. Fisheries Research \textbf{109}(1): 89--94.
doi:\href{https://doi.org/10.1016/j.fishres.2011.01.021}{10.1016/j.fishres.2011.01.021}.

\leavevmode\vadjust pre{\hypertarget{ref-legaultrestrepo99}{}}%
Legault, C.M., and Restrepo, V.R. 1999. A flexible forward
age-structured assessment program. Col. Vol. Sci. Pap. ICCAT
\textbf{49}(2): 246--253.

\leavevmode\vadjust pre{\hypertarget{ref-lietal24}{}}%
Li, C., Deroba, J.J., Miller, T.J., Legault, C.M., and Perretti, C.T.
2024. An evaluation of common stock assessment diagnostic tools for
choosing among state-space models with multiple random effects
processes. Fisheries Research \textbf{273}: 106968.
doi:\href{https://doi.org/10.1016/j.fishres.2024.106968}{10.1016/j.fishres.2024.106968}.

\leavevmode\vadjust pre{\hypertarget{ref-maunder24}{}}%
Maunder, M.N. 2024. Towards a comprehensive framework for providing
management advice from statistical inference using population dynamics
models. Ecological Modelling \textbf{498}: 110836.
doi:\href{https://doi.org/10.1016/j.ecolmodel.2024.110836}{10.1016/j.ecolmodel.2024.110836}.

\leavevmode\vadjust pre{\hypertarget{ref-methotwetzel13}{}}%
Methot, R.D., and Wetzel, C.R. 2013. Stock {S}ynthesis: A biological and
statistical framework for fish stock assessment and fishery management.
Fisheries Research \textbf{142}(1): 86--99.

\leavevmode\vadjust pre{\hypertarget{ref-milleretal16_yoy_survival}{}}%
Miller, A.S., Shepherd, G.R., and Fratantoni, P.S. 2016a. Offshore
habitat preference of overwintering juvenile and adult black sea bass,
\emph{{C}entropristis} \emph{striata}, and the relationship to
year-class success. {PLOS} {ONE} \textbf{11}(1): e0147627.
doi:\href{https://doi.org/10.1371/journal.pone.0147627}{10.1371/journal.pone.0147627}.

\leavevmode\vadjust pre{\hypertarget{ref-millerandersen08}{}}%
Miller, T.J., and Andersen, P.K. 2008. A finite-state continuous-time
approach for inferring regional migration and mortality rates from
archival tagging and conventional tag-recovery experiments. Biometrics
\textbf{64}(4): 1196--1206.
doi:\href{https://doi.org/10.1111/j.1541-0420.2008.00996.x}{10.1111/j.1541-0420.2008.00996.x}.

\leavevmode\vadjust pre{\hypertarget{ref-milleretal16}{}}%
Miller, T.J., Hare, J.A., and Alade, L.A. 2016b. A state-space approach
to incorporating environmental effects on recruitment in an
age-structured assessment model with an application to southern {New
England} yellowtail flounder. Can. J. Fish. Aquat. Sci. \textbf{73}(8):
1261--1270.
doi:\href{https://doi.org/10.1139/cjfas-2015-0339}{10.1139/cjfas-2015-0339}.

\leavevmode\vadjust pre{\hypertarget{ref-milleretal18}{}}%
Miller, T.J., O'Brien, L., and Fratantoni, P.S. 2018. Temporal and
environmental variation in growth and maturity and effects on management
reference points of {Georges Bank Atlantic} cod. Can. J. Fish. Aquat.
Sci. \textbf{75}(12): 2159--2171.
doi:\href{https://doi.org/10.1139/cjfas-2017-0124}{10.1139/cjfas-2017-0124}.

\leavevmode\vadjust pre{\hypertarget{ref-millerstock20}{}}%
Miller, T.J., and Stock, B.C. 2020. The {Woods Hole Assessment Model}
({WHAM}). https://timjmiller.github.io/wham.

\leavevmode\vadjust pre{\hypertarget{ref-mohn99}{}}%
Mohn, R. 1999. The retrospective problem in sequential population
analysis: An investigation using cod fishery and simulated data. ICES
Journal of Marine Science \textbf{56}(4): 473--488.

\leavevmode\vadjust pre{\hypertarget{ref-myers98}{}}%
Myers, R.A. 1998. When do environment--recruitment correlations work?
Reviews in Fish Biology and Fisheries \textbf{8}(3): 229--249.
doi:\href{https://doi.org/10.1023/A:1008828730759}{10.1023/A:1008828730759}.

\leavevmode\vadjust pre{\hypertarget{ref-nefsc22}{}}%
NEFSC. 2022a. Final report of the haddock research track assessment
working group. {Available} at
https://s3.us-east-1.amazonaws.com/nefmc.org/14b\_EGB\_Research\_Track\_Haddock\_WG\_Report\_DRAFT.pdf.

\leavevmode\vadjust pre{\hypertarget{ref-nefsc22a}{}}%
NEFSC. 2022b. Report of the american plaice research track working
group. {Available} at
https://s3.us-east-1.amazonaws.com/nefmc.org/2\_American-Plaice-WG-Report.pdf.

\leavevmode\vadjust pre{\hypertarget{ref-nefsc23}{}}%
NEFSC. 2023. Report of the black sea bass (\emph{{C}entropristis}
\emph{striata}) research track stock assessment working group.
{Available} at
https://www.mafmc.org/s/a\_2023\_BSB\_UNIT\_RTWG\_Report\_V2\_12\_2\_2023-1.pdf.

\leavevmode\vadjust pre{\hypertarget{ref-nefsc24}{}}%
NEFSC. 2024. Butterfish research track assessment report. US Dept Commer
Northeast Fish Sci Cent Ref Doc. 24-03; 191 p.

\leavevmode\vadjust pre{\hypertarget{ref-nielsenberg14}{}}%
Nielsen, A., and Berg, C.W. 2014. Estimation of time-varying selectivity
in stock assessments using state-space models. Fisheries Research
\textbf{158}: 96--101.
doi:\href{https://doi.org/10.1016/j.fishres.2014.01.014}{10.1016/j.fishres.2014.01.014}.

\leavevmode\vadjust pre{\hypertarget{ref-perreaultetal20}{}}%
Perreault, A.M.J., Wheeland, L.J., Morgan, M.J., and Cadigan, N.G. 2020.
A state-space stock assessment model for {American} plaice on the
{Grand} {Bank} of {Newfoundland}. Journal of Northwest Atlantic Fishery
Science \textbf{51}: 45--104.
doi:\href{https://doi.org/10.2960/j.v51.m727}{10.2960/j.v51.m727}.

\leavevmode\vadjust pre{\hypertarget{ref-puntetal20}{}}%
Punt, A.E., Dunn, A., Elvarsson, B., Hampton, J., Hoyle, S.D., Maunder,
M.N., Methot, R.D., and Nielsen, A. 2020. Essential features of the
next-generation integrated fisheries stock assessment package: A
perspective. Fisheries Research \textbf{229}: 105617.
doi:\url{https://doi.org/10.1016/j.fishres.2020.105617}.

\leavevmode\vadjust pre{\hypertarget{ref-schwarzetal93}{}}%
Schwarz, C.J., Schweigert, J.F., and Arnason, A.N. 1993. Estimating
migration rates using tag-recovery data. Biometrics \textbf{49}:
177--193. doi:\href{https://doi.org/10.2307/2532612}{10.2307/2532612}.

\leavevmode\vadjust pre{\hypertarget{ref-stockmiller21}{}}%
Stock, B.C., and Miller, T.J. 2021. The {Woods Hole Assessment Model}
({WHAM}): A general state-space assessment framework that incorporates
time- and age-varying processes via random effects and links to
environmental covariates. Fisheries Research \textbf{240}: 105967.
doi:\href{https://doi.org/10.1016/j.fishres.2021.105967}{10.1016/j.fishres.2021.105967}.

\leavevmode\vadjust pre{\hypertarget{ref-stocketal21}{}}%
Stock, B.C., Xu, H., Miller, T.J., Thorson, J.T., and Nye, J.A. 2021.
Implementing two-dimensional autocorrelation in either survival or
natural mortality improves a state-space assessment model for {Southern
New England}-{Mid Atlantic} yellowtail flounder. Fisheries Research
\textbf{237}: 105873.
doi:\href{https://doi.org/10.1016/j.fishres.2021.105873}{10.1016/j.fishres.2021.105873}.

\leavevmode\vadjust pre{\hypertarget{ref-sullivanetal05}{}}%
Sullivan, M.C., Cowen, R.K., and Steves, B.P. 2005. Evidence for
atmosphere{--}ocean forcing of yellowtail flounder (\emph{{L}imanda}
\emph{ferruginea}) recruitment in the {M}iddle {A}tlantic {B}ight.
Fisheries Oceanography \textbf{14}(5): 386--399.

\leavevmode\vadjust pre{\hypertarget{ref-thorsonetal17}{}}%
Thorson, J.T., Johnson, K.F., Methot, R.D., and Taylor, I.G. 2017.
Model-based estimates of effective sample size in stock assessment
models using the {D}irichlet-multinomial distribution. Fisheries
Research \textbf{192}: 84--93.
doi:\href{https://doi.org/10.1016/j.fishres.2016.06.005}{10.1016/j.fishres.2016.06.005}.

\leavevmode\vadjust pre{\hypertarget{ref-thorsonetal23}{}}%
Thorson, J.T., Miller, T.J., and Stock, B.C. 2023. The
multivariate-tweedie: A self-weighting likelihood for age and length
composition data arising from hierarchical sampling designs. ICES
Journal of Marine Science \textbf{80}(10): 2630--2641.
doi:\href{https://doi.org/10.1093/icesjms/fsac159}{10.1093/icesjms/fsac159}.

\leavevmode\vadjust pre{\hypertarget{ref-thygesenetal17}{}}%
Thygesen, U.H., Albertsen, C.M., Berg, C.W., Kristensen, K., and
Nielsen, A. 2017. {Validation of ecological state space models using the
Laplace approximation}. Environmental and Ecological Statistics
\textbf{24}(2): 317--339.
doi:\href{https://doi.org/10.1007/s10651-017-0372-4}{10.1007/s10651-017-0372-4}.

\leavevmode\vadjust pre{\hypertarget{ref-trijouletetal23}{}}%
Trijoulet, V., Albertsen, C.M., Kristensen, K., Legault, C.M., Miller,
T.J., and Nielsen, A. 2023. Model validation for compositional data in
stock assessment models: Calculating residuals with correct properties.
Fisheries Research \textbf{257}: 106487.
doi:\url{https://doi.org/10.1016/j.fishres.2022.106487}.

\leavevmode\vadjust pre{\hypertarget{ref-varkeyetal22}{}}%
Varkey, D.A., Babyn, J., Regular, P., Ings, D.W., Kumar, R., Rogers, B.,
Champagnat, J., and Morgan, M.J. 2022. A state-space model for stock
assessment of cod (gadus morhua) stock in NAFO subdivision 3Ps. {DFO}
{Can.} {Sci.} {Advis.} {Sec.} {Res.} {Doc.} 2022/022. v + 78 p.

\leavevmode\vadjust pre{\hypertarget{ref-yingetal11}{}}%
Ying, Y., Chen, Y., Lin, L., and Gao, T. 2011. Risks of ignoring fish
population spatial structure in fisheries management. Canadian Journal
of Fisheries and Aquatic Sciences \textbf{68}(12): 2101--2120.
doi:\href{https://doi.org/10.1139/f2011-116}{10.1139/f2011-116}.

\leavevmode\vadjust pre{\hypertarget{ref-zhengcadigan23}{}}%
Zheng, N., and Cadigan, N. 2023. Frequentist conditional variance for
nonlinear mixed-effects models. Journal of Statistical Theory and
Practice \textbf{17}(3).
doi:\href{https://doi.org/10.1007/s42519-022-00304-5}{10.1007/s42519-022-00304-5}.

\end{CSLReferences}

\pagebreak

\hypertarget{appendix-a}{%
\section*{Appendix A}\label{appendix-a}}
\addcontentsline{toc}{section}{Appendix A}

\setcounter{table}{0}
\renewcommand\thetable{A\arabic{table}}
\begin{longtable}[c]{r p{0.85\textwidth}}
\caption{Definition of terms.\label{symbols}}%
\\ \hline \hline \endfirsthead 
\caption[]{(Continued)} %
\\ \hline \hline \endhead
 \hline \hline \endfoot
$i$ & Seasonal time interval\\
$\delta_i$ & Length of seasonal time interval $i$\\
$a$ & Age class\\
$y$ & Year\\
$A$ & Last age class (``plus group'')\\
$r$ & Region\\
$f$ & Fishing fleet\\
$n_F$ & Number of fishing fleets\\
$s$ & Stock\\
$n_R$ & Number of regions\\
$\mathbf{P}_{y,a,i}$ & Probability transition matrix for year $y$, age $a$, and season $i$\\
$\mathbf{O}_{y,a,i}$ & submatrix of $\mathbf{P}_{y,a,i}$ of probabilities of surviving and occuring in each region for year $y$, age $a$, and season $i$\\
$\mathbf{H}_{y,a,i}$ & submatrix of $\mathbf{P}_{y,a,i}$ of probabilities of being captured in each fishing fleet for year $y$, age $a$, and season $i$\\
$\mathbf{I}_{H}$ & $n_f$ x $n_f$ identity matrix \\
$m$ & Index observation\\
$O_{y,a,i}(r,r')$ & For year $y$, age $a$ and season $i$, the probability of surviving and occurring in region $r'$ given beginning the interval alive in region $r$\\
$H_{y,a,i}(r,f)$ & For year $y$, age $a$ and season $i$, the probability of being captured in fleet $f$ given beginning the interval alive in region $r$\\
$S_{y,a,i}$ & For year $y$, age $a$ and season $i$, the probability of surviving the interval (1 region model)\\
$F_{y,a,i,f}$ & Fishing mortality rate for fleet $f$ in year $y$ at age $a$ in seasonal interval $i$\\
$M_{y,a}$ &  Natural mortality rate in year $y$ at age $a$ (single region)\\
$M_{y,a,r}$ & Natural mortality rate in region $r$ and year $y$ at age $a$\\
$Z_{y,a,i}$ & Total mortality rate in year $y$ at age $a$ in seasonal interval $i$ (single region)\\
$Z_{y,a,i,r}$ & Total mortality rate in region $r$ and year $y$ at age $a$ in seasonal interval $i$\\
$\mathbf{S}_{y,a,i}$ & matrix of of probabilities of surviving in each region over the interval for season $i$, year $y$, age $a$ \\
$\boldsymbol{\mu}_{y,a,i}$ & matrix of of probabilities of moving or staying in each region at the end of season $i$ in year $y$ and age $a$ \\
$\mu_{r\rightarrow r',y,a,i}$ & For year $y$, age $a$ and seasonal interval $i$, either the probability of moving at the end of the interval or instantanteous rate of movement from region $r$ to region $r'$\\
$r_f$ & region where fleet $f$ operates\\
$\mathbf{N}_{y,a}$ & Column vector of abundances by region at age $a$ in year $y$\\
$\mathbf{A}_{y,a,i}$ & instantaneous rate matrix for seasonal interval $i$, year $y$, and age $a$\\
$a_{y,a,i,r}$ & For year $y$, age $a$ and seasonal interval $i$, the hazard or negative sum of the instantaneous rates of mortality and movement from the state corresonding to being alive in region $r$ \\
$\mathbf{P}_{y,a}(\delta_1,\ldots,\delta_K)$ & Probability transition matrix for year $y$ and age $a$ over seasonal intervals $\delta_1,\ldots, \delta_K$\\
$K$ & Number of seasons in the annual time step\\
$t_s$ & fraction of the annual time step when spawning occurs for stock $s$\\
$\delta_{s,j}$ & fraction of the annual time step between $t_s$ and the end of season $j-1$\\
$t_m$ & fraction of the annual time step when index $m$ observers the population\\
$\delta_{m,j}$ & fraction of the annual time step between $t_m$ and the end of season $j-1$\\
$\mathbf{N}_{y,a}$ & Abundance at age $a$ in year $y$ in each of the living and mortality states on January 1\\
$\mathbf{N}_{O,y,a}$ & Abundance at age $a$ in year $y$ alive in each region on January 1\\
$N_{y,a,r}$ & Abundance at age $a$ in year $y$ alive in region $r$ on January 1\\
$r_s$ & region where stock $s$ spawns and recruits\\
$\text{SSB}_{s,y}$ & Spawning stock biomass for stock $s$ in year $y$\\
$\varepsilon_{y,a,r}$ & Random error for abundance at age $a$ in year $y$ in region $r$\\
$w_{s,y,a}$ & mean individual weight for stock $s$ at age $a$ in year $y$\\
$\text{mat}_{s,y,a}$ & proportion mature at age $a$ in year $y$ for stock $s$\\
$\mathbf{O}_{s,y,a,r_s}(t_s)$ & Probabilities of surviving and occurring in region $r_s$ at time $t_s$ given being alive in each region at the start of the year\\
$\boldsymbol{\varepsilon}_{y,a}$ & vector of random errors for abundance alive in each region on January 1 of year $y$ at age $a$\\
$\sigma_{N,r}$ & standard deviation parameter for abundance at age random effects in region $r$\\
$\rho_{N,\text{age},r}$ & first order auto-regressive correlation parameter across age for abundance at age random effects in region $r$\\
$\rho_{N,\text{year},r}$ & first order auto-regressive correlation parameter across year for abundance at age random effects in region $r$\\
$\text{sel}_{1,a,f}$ & selectivity at age $a$ for fleet $f$ in the first year\\
$F_{1,a,f}$ & fishing mortality rate at age $a$ for fleet $f$ in the first year\\
$\widetilde{F}_{a,f}$ & equilibrium fishing mortality rate at age $a$ for fleet $f$\\
$\mathbf{O}_{a}$ & proportion surviving the year at age $j$ and occuring in each region (columns) given alive on January 1 in each region (rows)\\
$\widetilde{\mathbf{O}}_{a}$ & equilibrium proportions alive in each region at age $a$ (columns) given recruitment in each region (rows)\\
$\mathbf{N}_{O,1,r}$ & vector of abundance by age in region $r$ in the first year\\
$\theta_{N_1,r}$ & mean parameter for initial numbers at age random effects in the first year for region $r$\\
$\boldsymbol{\varepsilon}_{N_1,r}$ & vector of random effects by age for initial numbers at age in the first year for region $r$\\ 
$\sigma_{N_1,r}$ & standard deviation parameter for initial numbers at age random effects in the first year for region $r$\\
$\rho_{N_1,r}$ & first order auto-regressive correlation parameter for initial numbers at age random effects in the first year for region $r$\\
$\mu_{r\rightarrow r',y,a,i}$ & movement from region $r$ to region $r'$ in seasonal interval $i$ and year $y$ at age $a$\\
$g(\mu_{r\rightarrow r',y,a,i})$ & link function for movement $\mu_{r\rightarrow r',y,a,i}$\\
$\theta_{r\rightarrow r',i}$ & mean parameter across age and year for movement from region $r$ to region $r'$ in seasonal interval $i$\\
$\varepsilon_{r\rightarrow r',y,a,i}$ &  random effect parameter for movement from region $r$ to region $r'$ in seasonal interval $i$ and year $y$ at age $a$\\
$n_E$ & number of environmental covariates\\
$\beta_{r \rightarrow r',a,i,k}$ & effect of environmental covariate $k$ on movement from region $r$ to $r'$ at age $a$ in seasonal interval $i$\\
$E_{k,y}$ & latent environmetal covariate $k$ affecting the population in year $y$\\
$\sigma_{r \rightarrow r',i}$ & standard deviation parameter for movement random effects from region $r$ to $r'$ in seasonal interval $i$\\
$\rho_{r \rightarrow r',\text{age},i}$ & first order auto-regressive correlation parameter across age for movement random effects from region $r$ to $r'$ in seasonal interval $i$\\
$\rho_{r \rightarrow r',\text{year},i}$ & first order auto-regressive correlation parameter across year for movement random effects from region $r$ to $r'$ in seasonal interval $i$\\
$\gamma_{r\rightarrow r',i}$ & random effect for link-transformed mean movement from region $r$ to $r'$ in seasonal interval $i$ when a prior distribution is assumed\\
$\sigma_{r\rightarrow r',i}$ & standard deviation parameter for prior distribution of $\gamma_{r\rightarrow r',i}$\\
$M_{y,a,r}$ & natural mortality rate for age $a$ in year $y$ in region $r$\\
$\theta_{M,r}$ & mean parameter across age and year for nutural mortality in region $r$\\
$\varepsilon_{M,r,y,a}$ & random effect parameter for natural mortality in region $r$ and year $y$ at age $a$\\
$\beta_{M,r,a,k}$ & effect of environmental covariate $k$ on natural mortality in region $r$ at age $a$\\
$\sigma_{M,r}$ & standard deviation parameter for natural mortality random effects in region $r$\\
$\rho_{M,\text{age},r}$ & first order auto-regressive correlation parameter across age for natural mortality random effects in region $r$\\
$\rho_{M,\text{year},r}$ & first order auto-regressive correlation parameter across year for natural mortality random effects in region $r$\\
$\widehat{\mathbf{N}}_{H,s,y,a}$ & vector of predicted numbers of stock $s$ at age $a$ in year $y$ captured by each fleet\\
$\widehat{\mathbf{N}}_{H,y,a}$ & vector of predicted numbers at age $a$ in year $y$ captured by each fleet across all stocks\\
$\widehat{\mathbf{C}}_{y,a}$ & vector of predicted biomass captured at age $a$ in year $y$ by each fleet across all stocks\\
$\mathbf{c}_{y,a}$ & vector of mean individual weight at age $a$ in year $y$ for each fleet\\
$\widehat{\mathbf{C}}_y$ & vector of predicted aggregate catch for each fleet in year $y$\\
$C_{y,f}$ & observed aggregate catch for fleet $f$ in year $y$\\
$\widehat C_{y,f}$ & predicted aggregate catch for fleet $f$ in year $y$\\
$\sigma_{y,f}$ & standard deviation of observed log-aggregate catch for fleet $f$ in year $y$\\
$\widehat{N}_{s,y,a,m}$ & predicted abundance at $t_m$ in region $r_m$\\
$\mathbf{O}_{s,y,a,r_m}(t_m)$ & the probabilities of surviving and occurring in region $r_m$ at time $t_m$ given being alive in each region at the start of the year which is the $r_m$ column of the upper-left submatrix of Eq. \ref{eq:ptm_index}\\
$\widehat I_{m,y,a}$ & Predicted relative abundance index for survey $d$ in year $y$ at age $a$ \\
$q_{m,y}$ & catchability of index $m$ in year $y$\\
$\text{sel}_{m,y,a}$ &  selectivity of index $m$ at age $a$ in year $y$ \\
$w_{m,y,a}$ & average weight of individuals at age $a$ for index $m$ if the index is quantified in biomass, otherwise it is unity\\
$u_{m}$ & upper bound for index $m$ catchability\\
$l_{m}$ & lower bound for index $m$ catchability\\
$\theta_{q,m}$ & mean index $m$ catchability parameter\\
$\varepsilon_{q,m,y}$ & index $m$ catchability random effect in year $y$\\
$\beta_{q,m,k}$ & effect of environmental covariate $k$ on index $m$ catchability\\
$\sigma_{q,r}$ & standard deviation parameter for index $m$ catchability random effects\\
$\rho_{q,m}$ & first order auto-regressive correlation parameter across year for index $m$ catchability random effects\\
\end{longtable}


\pagebreak

\hypertarget{figures}{%
\section*{Figures}\label{figures}}
\addcontentsline{toc}{section}{Figures}

\begin{figure}

{\centering \includegraphics[width=0.8\linewidth]{bsb_movement_diagram} 

}

\caption{Diagram of intervals within the year and configuation of the dynamics of each component of the BSB population.}\label{fig:migration-diagram}
\end{figure}
\pagebreak

\begin{figure}

{\centering \includegraphics[width=1\linewidth]{move_prior_post} 

}

\caption{Prior (black) and posterior (red) distributions of movement of northern component stock from north to south (top) and south to north (bottom).}\label{fig:move-prior-posterior}
\end{figure}
\pagebreak

\begin{figure}

{\centering \includegraphics[width=1\linewidth]{Ecov} 

}

\caption{Observations and 95\% confidence intervals (points with vertical lines) and posterior estimates (lines) with 95\% confidence intervals (polygons) of bottom temperature anomalies in the north and south regions from model $M_0$.}\label{fig:bottom-temperature}
\end{figure}
\pagebreak

\begin{figure}

{\centering \includegraphics[width=1\linewidth]{best_R_Ecov} 

}

\caption{Expected and Posterior Emperical Bayes estimates of annual recruitment for the northern stock component. Color defined by the corresponding annual bottom temperature anomaly.}\label{fig:BT-Ecov-R}
\end{figure}
\pagebreak

\begin{figure}

{\centering \includegraphics[height=0.95\textheight]{M_1_SSB_F_R} 

}

\caption{Annual estimates of SSB, average F (ages 6-7), and recruitment from model $M_1$. Polygons represent 95\% confidence intervals.}\label{fig:M1-SSB-F-R}
\end{figure}
\pagebreak

\begin{figure}

{\centering \includegraphics[height=0.95\textheight]{R_proj_results} 

}

\caption{Estimates of bottom temperature, recruitment, and expected recruitment, and coefficients of variation for the latter two from model $M_1$ assuming alternative bottom temperature scenarios described in the text. Polygons represent 95\% confidence intervals.}\label{fig:R-BT-proj}
\end{figure}

\pagebreak

\begin{figure}

{\centering \includegraphics[height=0.95\textheight]{selectivity_re_plot} 

}

\caption{Time and age-varying selectivty for fleets and indices in the northern region with autoregressive random effects.}\label{fig:selectivity-re}
\end{figure}
\pagebreak

\clearpage

\hypertarget{tables}{%
\section*{Tables}\label{tables}}
\addcontentsline{toc}{section}{Tables}

\setcounter{table}{0}
\renewcommand\thetable{\arabic{table}}

\begin{table}

\caption{\label{tab:model-desc-table}Assumptions for temperature effects and random effects for age 1 natural mortality for each model.}
\centering
\begin{tabular}[t]{llll}
\toprule
\multicolumn{1}{c}{ } & \multicolumn{2}{c}{Temperature Effect} & \multicolumn{1}{c}{ } \\
\cmidrule(l{3pt}r{3pt}){2-3}
Model & North & South & $M$ at age 1 random effects\\
\midrule
$M_{0}$ & -- & -- & none\\
$M_{1}$ & Recruitment & -- & none\\
$M_{2}$ & -- & Recruitment & none\\
$M_{3}$ & Recruitment & Recruitment & none\\
$M_{4}$ & $M$ at age 1 & -- & none\\
\addlinespace
$M_{5}$ & -- & $M$ at age 1 & none\\
$M_{6}$ & $M$ at age 1 & $M$ at age 1 & none\\
$M_{7}$ & -- & -- & time-varying\\
$M_{8}$ & Recruitment & -- & time-varying\\
$M_{9}$ & -- & Recruitment & time-varying\\
\addlinespace
$M_{10}$ & Recruitment & Recruitment & time-varying\\
$M_{11}$ & $M$ at age 1 & -- & time-varying\\
$M_{12}$ & -- & $M$ at age 1 & time-varying\\
$M_{13}$ & $M$ at age 1 & $M$ at age 1 & time-varying\\
\bottomrule
\end{tabular}
\end{table}

\begin{table}

\caption{\label{tab:diff-aic-table}Difference between AIC and the lowest AIC for each model by retrospective peel.}
\centering
\begin{tabular}[t]{lrrrrrrrr}
\toprule
\multicolumn{1}{c}{ } & \multicolumn{8}{c}{Peel} \\
\cmidrule(l{3pt}r{3pt}){2-9}
Model & 0 & 1 & 2 & 3 & 4 & 5 & 6 & 7\\
\midrule
$M_{0}$ & 11.83 & 11.39 & 10.41 & 10.05 & 9.86 & 9.41 & 8.47 & 8.22\\
$M_{1}$ & 0.00 & 0.00 & 0.00 & 0.00 & 0.00 & 0.00 & 0.00 & 0.00\\
$M_{2}$ & 12.63 & 12.03 & 11.41 & 11.06 & 10.91 & 10.34 & 9.20 & 9.22\\
$M_{3}$ & 0.80 & 0.64 & 1.00 & 1.01 & 1.05 & 0.93 & 0.73 & 1.00\\
$M_{4}$ & 13.81 & 13.35 & 12.26 & 11.82 & 11.71 & 11.18 & 10.08 & 9.85\\
\addlinespace
$M_{5}$ & 13.25 & 12.55 & 11.68 & 11.21 & 11.07 & 10.41 & 9.86 & 9.71\\
$M_{6}$ & 15.22 & 14.51 & 13.52 & 12.97 & 12.91 & 12.17 & 11.46 & 11.33\\
$M_{7}$ & 14.32 & 13.75 & 12.17 & 11.30 & 11.43 & 10.55 & 9.71 & 9.42\\
$M_{8}$ & 2.25 & 2.10 & 1.47 & 0.96 & 1.29 & 0.91 & 0.98 & 0.99\\
$M_{9}$ & 15.12 & 14.39 & 13.17 & 12.31 & 12.48 & 11.49 & 10.44 & 10.43\\
\addlinespace
$M_{10}$ & 3.05 & 2.74 & 2.47 & 1.97 & 2.34 & 1.84 & 1.71 & 1.99\\
$M_{11}$ & 16.29 & 15.74 & 14.17 & 13.30 & 13.43 & 12.55 & 11.68 & 11.39\\
$M_{12}$ & 15.73 & 14.91 & 13.43 & 12.46 & 12.64 & 11.55 & 11.10 & 10.92\\
$M_{13}$ & 17.70 & 16.91 & 15.43 & 14.46 & 14.64 & 13.55 & 13.07 & 12.88\\
\bottomrule
\end{tabular}
\end{table}

\begin{table}

\caption{\label{tab:aic-wts-table}Model AIC weights for each retrospective peel.}
\centering
\begin{tabular}[t]{lrrrrrrrr}
\toprule
\multicolumn{1}{c}{ } & \multicolumn{8}{c}{Peel} \\
\cmidrule(l{3pt}r{3pt}){2-9}
Model & 0 & 1 & 2 & 3 & 4 & 5 & 6 & 7\\
\midrule
$M_{0}$ & 0.00 & 0.00 & 0.00 & 0.00 & 0.00 & 0.00 & 0.01 & 0.01\\
$M_{1}$ & 0.45 & 0.43 & 0.42 & 0.38 & 0.41 & 0.37 & 0.36 & 0.38\\
$M_{2}$ & 0.00 & 0.00 & 0.00 & 0.00 & 0.00 & 0.00 & 0.00 & 0.00\\
$M_{3}$ & 0.30 & 0.31 & 0.25 & 0.23 & 0.24 & 0.23 & 0.25 & 0.23\\
$M_{4}$ & 0.00 & 0.00 & 0.00 & 0.00 & 0.00 & 0.00 & 0.00 & 0.00\\
\addlinespace
$M_{5}$ & 0.00 & 0.00 & 0.00 & 0.00 & 0.00 & 0.00 & 0.00 & 0.00\\
$M_{6}$ & 0.00 & 0.00 & 0.00 & 0.00 & 0.00 & 0.00 & 0.00 & 0.00\\
$M_{7}$ & 0.00 & 0.00 & 0.00 & 0.00 & 0.00 & 0.00 & 0.00 & 0.00\\
$M_{8}$ & 0.15 & 0.15 & 0.20 & 0.24 & 0.21 & 0.24 & 0.22 & 0.23\\
$M_{9}$ & 0.00 & 0.00 & 0.00 & 0.00 & 0.00 & 0.00 & 0.00 & 0.00\\
\addlinespace
$M_{10}$ & 0.10 & 0.11 & 0.12 & 0.14 & 0.13 & 0.15 & 0.15 & 0.14\\
$M_{11}$ & 0.00 & 0.00 & 0.00 & 0.00 & 0.00 & 0.00 & 0.00 & 0.00\\
$M_{12}$ & 0.00 & 0.00 & 0.00 & 0.00 & 0.00 & 0.00 & 0.00 & 0.00\\
$M_{13}$ & 0.00 & 0.00 & 0.00 & 0.00 & 0.00 & 0.00 & 0.00 & 0.00\\
\bottomrule
\end{tabular}
\end{table}

\begin{table}

\caption{\label{tab:rho-table}Mohn's $\rho$ for SSB, and average F at ages 6 and 7 in northern and southern regions.}
\centering
\begin{tabular}[t]{lrrrr}
\toprule
\multicolumn{1}{c}{ } & \multicolumn{2}{c}{SSB} & \multicolumn{2}{c}{Average F} \\
\cmidrule(l{3pt}r{3pt}){2-3} \cmidrule(l{3pt}r{3pt}){4-5}
Model & North & South & North & South\\
\midrule
$M_{0}$ & -0.040 & -0.023 & 0.041 & -0.048\\
$M_{1}$ & -0.040 & -0.023 & 0.041 & -0.048\\
$M_{2}$ & -0.040 & -0.024 & 0.041 & -0.047\\
$M_{3}$ & -0.040 & -0.024 & 0.041 & -0.047\\
$M_{4}$ & -0.041 & -0.022 & 0.042 & -0.048\\
\addlinespace
$M_{5}$ & -0.041 & -0.023 & 0.042 & -0.048\\
$M_{6}$ & -0.042 & -0.023 & 0.042 & -0.048\\
$M_{7}$ & -0.042 & -0.022 & 0.043 & -0.048\\
$M_{8}$ & -0.044 & -0.023 & 0.044 & -0.048\\
$M_{9}$ & -0.042 & -0.024 & 0.043 & -0.048\\
\addlinespace
$M_{10}$ & -0.044 & -0.024 & 0.044 & -0.047\\
$M_{11}$ & -0.041 & -0.022 & 0.042 & -0.048\\
$M_{12}$ & -0.042 & -0.023 & 0.043 & -0.048\\
$M_{13}$ & -0.041 & -0.023 & 0.042 & -0.048\\
\bottomrule
\end{tabular}
\end{table}
\begin{table}

\caption{\label{tab:beta-sig-peel-table}Estimates of temperature effects on recruitment and variance and autocorrelation parameters for recruitment for northern ($N$) and southern ($S$) components for models with no effects ($M_0$), effects on specific components ($M_1$ and $M_2$) or both components simultaneously ($M_3$) from the full model (Peel 0) and each retrospective peel.}
\centering
\begin{tabular}[t]{lrrrrrrrr}
\toprule
\multicolumn{1}{c}{ } & \multicolumn{8}{c}{Peel} \\
\cmidrule(l{3pt}r{3pt}){2-9}
Parameter & 0 & 1 & 2 & 3 & 4 & 5 & 6 & 7\\
\midrule
$M_{1}$ $\widehat{\beta}_{R,N}$ & 0.474 & 0.480 & 0.485 & 0.476 & 0.464 & 0.468 & 0.439 & 0.445\\
$M_{2}$ $\widehat{\beta}_{R,S}$ & 0.099 & 0.105 & 0.094 & 0.095 & 0.092 & 0.096 & 0.099 & 0.091\\
$M_{3}$ $\widehat{\beta}_{R,N}$ & 0.474 & 0.480 & 0.485 & 0.476 & 0.464 & 0.468 & 0.439 & 0.445\\
$M_{3}$ $\widehat{\beta}_{R,S}$ & 0.099 & 0.105 & 0.094 & 0.095 & 0.092 & 0.096 & 0.099 & 0.091\\
$M_{0}$ Conditional $\widehat{\sigma}_{R,N}$ & 0.925 & 0.953 & 0.978 & 0.988 & 0.978 & 1.010 & 0.981 & 1.007\\
\addlinespace
$M_{1}$ Conditional $\widehat{\sigma}_{R,N}$ & 0.730 & 0.752 & 0.780 & 0.786 & 0.775 & 0.801 & 0.785 & 0.800\\
$M_{0}$ $\widehat{\rho}_{R,N}$ & 0.362 & 0.375 & 0.385 & 0.405 & 0.424 & 0.432 & 0.427 & 0.431\\
$M_{1}$ $\widehat{\rho}_{R,N}$ & 0.296 & 0.307 & 0.334 & 0.373 & 0.394 & 0.406 & 0.428 & 0.429\\
$M_{0}$ Marginal $\widehat{\sigma}_{R,N}$ & 0.992 & 1.028 & 1.060 & 1.080 & 1.081 & 1.120 & 1.084 & 1.117\\
$M_{1}$ Marginal $\widehat{\sigma}_{R,N}$ & 0.764 & 0.791 & 0.827 & 0.848 & 0.843 & 0.877 & 0.869 & 0.885\\
\bottomrule
\end{tabular}
\end{table}

\clearpage

\hypertarget{supplemental-materials}{%
\section*{Supplemental Materials}\label{supplemental-materials}}
\addcontentsline{toc}{section}{Supplemental Materials}

\hypertarget{deriving-the-prior-distribution-for-movement-parameters}{%
\subsection*{Deriving the prior distribution for movement
parameters}\label{deriving-the-prior-distribution-for-movement-parameters}}
\addcontentsline{toc}{subsection}{Deriving the prior distribution for
movement parameters}

The working group fit a Stock Synthesis model (Methot and Wetzel 2013)
that included tagging data with 2 seasons (6 months each) and 2 regions
where a proportion \(\mu^*_1\) of the northern component moves to the
south in one season and some proportion \(\mu^*_{2\rightarrow 1}\) move
back to the south in the second season (NEFSC 2023). The seasonal
movement matrices for each season are \begin{equation*}
\boldsymbol{\mu}^*_{1} = 
  \begin{bmatrix}
     1-\mu^*_{1\rightarrow 2} & \mu^*_{1\rightarrow 2} \\
     0 & 1 \\
  \end{bmatrix}
\end{equation*} and \begin{equation*}
\boldsymbol{\mu}_{2} = 
  \begin{bmatrix}
     1 &  0 \\
     \mu^*_{2\rightarrow 1} & 1-\mu^*_{2\rightarrow 1} \\
  \end{bmatrix}.
\end{equation*}

To obtain estimates of movement proportions for the monthly intervals in
the WHAM model, the half-year movement matrices were converted to
monthly movement matrices by taking the root \(z_k\) of
\(\boldsymbol{\mu}^*_{k}\) which are defined by the number of months of
movement for each season (5 and 4, respectively). The roots of the
matrices are calculated using an eigen decomposition of the matrices
\[ \boldsymbol{\mu}_k =  \left(\boldsymbol{\mu}_k^*\right)^{z_k} = \mathbf{V}_k \mathbf{D}_k^{z_k} \mathbf{V}_k^{-1}\]
where \(z_1 = 1/5\) for and \(z_2 = 1/4\), and \(\mathbf{V}_{k}\) and
\(\mathbf{D}_{k}\) are the matrix of eigenvectors (columnwise) and the
diagonal matrix of corresponding eigenvalues of
\(\boldsymbol{\mu}^*_k\). The working group used a parametric bootstrap
approach to determine an appropriate standard deviation for the prior
distribution for the movement parameters. Stock Synthesis also estimates
parameters on a transformed scale, but different from WHAM:
\[\mu^*_{r\rightarrow r'} = \frac{1}{1 + 2e^{-x_{r\rightarrow r'}}}\]
The estimated parameters and standard errors from the Stock Synthesis
model were \(x_{1\rightarrow 2}=-1.44\) and \(x_{2\rightarrow 1}=1.94\)
and \(SE(x_{1\rightarrow 2})) = 0.21\) and
\(SE(x_{2\rightarrow 1})) = 0.37\). The resulting in the estimated
proportions were \(\mu^*_{1\rightarrow 2}=0.11\) and
\(\mu^*_{2\rightarrow 1}=0.78\).

In WHAM, an additive logit transformation is used which is simply a
logit transformation when there are only two regions: \[
\mu_{r\rightarrow r'} = \frac{1}{1+e^{-y_{r\rightarrow r'}}}.
\] We simulated 1000 values from a normal distribution with mean and
standard deviation defined by the parameter estimate and standard error
\(\tilde x_{{r\rightarrow r'},b} \sim N(x_{r\rightarrow r'}, SE(x_{r\rightarrow r'}))\)
from the Stock Synthesis model. For each simulated value we constructed
\(\tilde {\boldsymbol{\mu}}^*_{{r\rightarrow r'},b}\), took the
appropriate root and calculated inverse logit for
\(\tilde y_{{r\rightarrow r'},b}\). We calculated the mean and standard
deviation of the values \(y_{i,b}\). The mean values did not differ
meaningfully from the transformation of the original estimates
(\(y_{1\rightarrow 2} = -3.79\) and \(y_{2\rightarrow 1} = -0.79\)) and
the standard deviation was approximately 0.2 for both parameters.

\hypertarget{diagnostics}{%
\subsection*{Diagnostics}\label{diagnostics}}
\addcontentsline{toc}{subsection}{Diagnostics}

\hypertarget{jitter-fits-for-model-m_0}{%
\subsubsection{\texorpdfstring{Jitter fits for model
\(M_0\)}{Jitter fits for model M\_0}}\label{jitter-fits-for-model-m_0}}

WHAM by default completes three newton steps after the stats::nlminb
minimization function completes to reduce the gradient at the minimized
NLL. However, this generally has negligible effects on model estimates
and the NLL. To reduce computation time, we did not complete these
newton steps when performing jitter fits of the model. Without the
Newton steps, the maximum (absolute) gradient sizes are generally less
than 0.01 for models that converge satisfactorily.

The 50 jitter fits demonstrated that a local minimum was obtained for
the original fit of model \(M_0\) (Figure \ref{fig:jitter-M0}). Some
lower NLLs were obtained with unacceptable gradients, but a slightly
lower NLL was found with a satisfactory gradient and with and invertible
hessian. We therefore refit model \(M_0\) and all remaining models using
the better parameter estimates as initial values.

\hypertarget{jitter-fits-for-model-m_1}{%
\subsubsection{\texorpdfstring{Jitter fits for model
\(M_1\)}{Jitter fits for model M\_1}}\label{jitter-fits-for-model-m_1}}

The 50 jitter fits gave no evidence of a better minimization of the NLL.
Three lower NLLs were obtained, but with unacceptably large gradients
(Figure \ref{fig:jitter-M1}). The largest differences in parameter
estimates for these three jitters were for numbers at age and
selectivity random effects variance and correlation parameters.

\hypertarget{self-test-for-model-m_1}{%
\subsubsection{\texorpdfstring{Self test for model
\(M_1\)}{Self test for model M\_1}}\label{self-test-for-model-m_1}}

Initial fits to simulated data from model \(M_1\) showed estimation of
the observation error standard deviation multiplier for the recreational
catch-per angler indices in the north and south regions was unstable.
Many of the fits to the simulated data produced implausible estimates at
the 0 boundary for these parameters (very negative values on log-scale).
Therefore, we completed self-tests with these parameters fixed at the
true values.

For 7 of the the simulated data sets the model failed to optimize. The
maximum absolute gradient was \(<10^{-6}\) for only 9 and \(<10^{-4}\)
for 52 of the 93 successfully fitted models. The poor convergence
appeared to be attributable to the estimation of the scalar for the
standard errors of the log-transformed northern Recreational CPA index
for which estimates tended to 0 for nearly all of the fits (\(<0.01\)
for 83 fits). However, even across all fits including those with poor
convergence, the SSB estimates appeared to be reliable (Figure
\ref{fig:self-test-fig}).

\hypertarget{one-step-ahead-residuals-for-model-m_1}{%
\subsubsection{\texorpdfstring{One-step-ahead residuals for model
\(M_1\)}{One-step-ahead residuals for model M\_1}}\label{one-step-ahead-residuals-for-model-m_1}}

One-step-ahead residuals can now be calculated for all index, catch, and
environmental covariate observations using methods described by Thygesen
et al. (2017) and Trijoulet et al. (2023).

\setcounter{table}{0}
\renewcommand\thetable{S\arabic{table}}

\begin{landscape}\begin{table}

\caption{\label{tab:age-comp-sel-table}Configuration of age composition likelihoods, mean selectivity models, and selectivity random effects models for each age composition data component. For all logistic-normal likelihoods, any ages observed as zeros are treated as missing.}
\centering
\begin{tabular}[t]{llll}
\toprule
Data component & Age Composition Likelihood & Mean Selectivity model & Random effects Model\\
\midrule
North commercial fleet & Dirichlet-Multinomial & age-specific (ages > 3 fully selected) & AR1 correlation by age and year\\
North recreational fleet & Logistic-normal (Independent) & age-specific (ages > 6 fully selected) & AR1 correlation by age and year\\
South commercial fleet & Logistic-normal (AR1 correlation) & logistic & None\\
South recreational fleet & Logistic-normal (AR1 correlation) & logistic & None\\
North recreational CPA index & Logistic-normal (Independent) & age-specific (ages > 1 fully selected) & AR1 correlation by year\\
\addlinespace
North VAST index & Dirichlet-Multinomial & age-specific (ages > 4 fully selected) & AR1 correlation by age and year\\
South recreational CPA index & Logistic-normal (AR1 correlation) & age-specific (ages > 2 fully selected) & None\\
South VAST index & Logistic-normal (AR1 correlation) & age-specific (ages > 1 fully selected) & None\\
\bottomrule
\end{tabular}
\end{table}
\end{landscape}

\setcounter{figure}{0}
\renewcommand\thefigure{S\arabic{figure}}

\pagebreak

\begin{figure}

{\centering \includegraphics[width=1\linewidth]{fit_0_jitter_plt} 

}

\caption{Minimized negative log-likelihood for 50 fits where minimization used initial parameter values jittered from those provided by an initial fit for model $M_0$. Black jitters had maximum absolute gradient values < $10^{-10}$, grey jitters had values > $10^{-10}$ and < 1, and red jitters had values > 1.}\label{fig:jitter-M0}
\end{figure}
\pagebreak

\begin{figure}

{\centering \includegraphics[width=1\linewidth]{fit_1_jitter_plt} 

}

\caption{Minimized negative log-likelihood for 50 fits where minimization used initial parameter values jittered from those provided by an initial fit for model $M_1$. Fits with black dots had maximum absolute gradient value < 0.01 and fits with red dots had values > 10.}\label{fig:jitter-M1}
\end{figure}
\pagebreak

\begin{landscape}
 
\begin{figure}

{\centering \includegraphics[width=1\linewidth]{self_test_results} 

}

\caption{Median relative bias of SSB for the North and South stock components for estimation models where the observation variance of log-indices are fixed and estimation is by maximum marginal likelihood (ML) or Restricted Maximum Likelihood (REML) or where the standard deviation of logistic-normal age composition observations are assumed to be 0.1 (over sqrt(1000)) (Low Error). Black and Red dashed lines represent the median of the yeaerly medians and the median across all yearly relative errors, respectively.}\label{fig:self-test-fig}
\end{figure}
\end{landscape}

\begin{figure}

{\centering \includegraphics[height=0.95\textheight]{SSB_F_R_rel_M1} 

}

\caption{Estimates of SSB, F, and recruitment relative to those of the best performing model, $M_1$.}\label{fig:SSB-F-R-rel-M1}
\end{figure}

\begin{figure}

{\centering \includegraphics[height=0.95\textheight]{Ecov_M1_rel_M0} 

}

\caption{Relative differences in posterior estimates of northern region bottom temperature anomalies ($\widehat X$) from the null model without effects on recruitment ($M_0$) and with effects on the northern stock component ($M_1$).}\label{fig:Ecov-M1-rel-M0}
\end{figure}

\pagebreak

\end{document}
